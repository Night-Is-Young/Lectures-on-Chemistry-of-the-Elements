\documentclass{ctexart}
\usepackage{EC}
\begin{document}
\section{铊及其化合物}
\subsection{单质铊}
\subsubsection{铊元素的发现,分布与铊单质的制备}
和\ce{Ga}和\ce{In}一样,\ce{Tl}也是通过光谱发现的,它的名字来自其火焰光谱中的特征亮绿谱线,希腊语为$\theta\alpha\lambda\lambda\acute{o}\varsigma$,即英语中的\textit{thallos},意为嫩芽或嫩枝\footnote{讽刺的是,与如此富有生机的名字相反的是\ce{Tl}化合物具有强烈的毒性.}.\\
\indent 我们前面说过,IIIA族的较重的元素通常以共生临近元素的方式存在于自然界中.\ce{Tl}通常与\ce{Pb}共生在方铅矿中.因此,同样可以从焙烧各种硫化物矿的烟道灰中收集\ce{Tl}元素.将含有各种元素的烟道灰溶解于热的稀酸中,然后沉淀出不溶的\ce{PbSO4},再加\ce{HCl}沉淀出\ce{TlCl}.随后将其转化为\ce{Tl2SO4}后电解即可得到\ce{Tl}单质.
\subsubsection{铊单质的性质}
在各方面上,\ce{Tl}的性质都和\ce{Ag}和碱金属比较相似.单质铊非常软,可延展性很高,在室温下可以用刀切割.它具有金属光泽,但在接触空气之后,会变为蓝灰色.\\
\indent 长期置于空气中的铊会形成较厚的氧化物层.因此,保存时可以像碱金属一样可以将其浸泡在油里.\\
\indent 铊单质可以与水反应生成可溶的强碱\ce{TlOH}:
\begin{center}
    \ce{2Tl + 2H2O -> H2 + 2TlOH}
\end{center}
也可与\ce{H2SO4}或\ce{HNO3}等常见的酸反应生成\ce{Tl^I}的可溶盐.然而,与\ce{HCl}的反应会使其表面覆盖一层难溶的\ce{TlCl}沉淀.在其他方面,\ce{Tl^I}与极化程度更高的\ce{Ag+}更为相似,例如铬酸盐,硫化物的颜色与不溶性等等.
\subsection{铊的卤化物}
\subsubsection{铊的三卤化物及相关的配合物}
和这一周期的其它主族元素一样,\ce{Tl^{III}}具有强烈的氧化性.因此,铊的三卤化物比IIIA族轻金属的三卤化物稳定得多,并且
在化学上与它们大不相同.
\paragraph{三氟化铊\ce{TlF3}及其配合物}
\ce{TlF3}是这类化合物中唯一比较稳定的.\ce{TlF3}是白色晶体,但与\ce{AlF3}采取不同的结构,而与\ce{YF3}和$\beta$-\ce{BiF3}的结构比较类似.\\
\indent \ce{TlF3}不形成水合物,但对湿气很敏感,遇水即发生水解:
\begin{center}
    \ce{2TlF3 + 3H2O -> Tl2O3 + 6HF}
\end{center}

\indent 它在水溶液中也不形成\ce{[TlF4]-},在液态\ce{HF}中才能形成这一四面体形的配离子.\ce{NaTlF4}和\ce{LiTlF4}事实上具有类似萤石的构型;\ce{Na3TlF6}则与冰晶石结构类似.
\paragraph{三氯化铊\ce{TlCl3}和三溴化铊\ce{TlBr3}及其配合物}
我们前面说过,\ce{Tl^{III}}具有较强的氧化性,因此\ce{TlCl3}和\ce{TlBr3}都是不稳定的化合物,容易发生分解:
\begin{center}
    \ce{TlX3 -> TlX + X2}\ \ \ $(\ce{X}=\ce{Cl},\ce{Br})$
\end{center}

\indent 四面体形的\ce{[TlX4]-}\footnote{本小节中$\ce{X}=\ce{Cl},\ce{Br}.$}和八面体形的\ce{[TlX6]^3-}都已经通过将对应卤素的盐与\ce{TlX3}反应制得.四方锥形的\ce{[TlCl5]^2-}和双核的\ce{[Tl2Cl9]^3-}也已经被制得,后者是两个八面体共面连接的结构,在双核配合物中比较典型.
\bichemfig{TlCl52-}{1}{\ce{[TlCl5]^2-}的结构}{Tl2Cl93-}{1}{\ce{[Tl2Cl9]^3-}的结构}{\ce{Tl^{III}}的\ce{Cl-}配合物的结构}
\paragraph{三碘化铊\ce{TlI3}}
蒸发等物质的量的\ce{TlI}和\ce{I2}的\ce{HI}水溶液即可得到黑色的\ce{TlI3}晶体.\\
\indent \ce{TlI3}中的\ce{Tl}并非$+3$价.事实上,它与\ce{CsI3},\ce{NH4I3}等同晶,其中含有线形的\ce{[I3]-}离子和$+1$价的\ce{Tl+}离子.令人惊奇的是,在有过量\ce{I-}存在时,\ce{Tl^{III}}可以借由形成四面体配合物而稳定:
\begin{center}
    \ce{Tl[I3](s) + I-(aq) -> [Tl^{III}I4]-(aq)}
\end{center}

\indent \ce{TlI3}在碱性条件下不稳定,容易生成\ce{Tl2O3}沉淀:
\begin{center}
    \ce{2TlI3 + 6OH- -> Tl2O3 + 6I- + 3H2O}
\end{center}
这可能是因为\ce{Tl2O3}的溶解度极小的缘故.此外,在碱性条件下\ce{I3-}的氧化性也不算弱.
\subsubsection{铊的一卤化物及相关的配合物}
就卤化物而言,\ce{Tl^I}是铊最稳定的氧化态.这些卤化物的性质与银的卤化物比较相似.例如,除了\ce{TlF}极易溶于水之外,其它三种一卤化铊都难溶于水.\\
\indent 这些盐都可以作为\ce{X-}供体,生成类似\ce{Tl[GaF4]},\ce{Tl[AlCl4]}等盐.\\
\indent 氯化亚铊\ce{TlCl}与\ce{AgCl}相似,一经曝光就会变暗(高纯的\ce{TlCl}似乎不受光的作用):
\begin{center}
    \ce{2TlCl ->T[h$\nu$] 2Tl + Cl2}
\end{center}
\indent 溴化亚铊\ce{TlBr}见光亦分解.它能被浓\ce{HNO3}氧化为\ce{Tl2Br3},实际的组成为\ce{Tl3[TlBr6]}.\\
\indent 碘化亚铊\ce{TlI}具有类似\ce{NaCl}的结构,只是发生了难以言说的畸变.
\bichemfig{TlI-1}{0.1}{\ce{TlI}的晶胞示意图}{TlI-2}{0.1}{\ce{TlI}的晶胞沿$a$轴的投影图}{\ce{TlI}的晶体结构}
\subsubsection{铊的其它低价卤化物}
和本族较轻的元素一样,\ce{Tl}也存在\ce{Tl[TlCl4]},\ce{Tl[TlBr4]}等混合价态的卤化物.一个有趣的例子是\ce{Tl2O3}与\ce{NH3}反应生成\ce{TlN6}(这一反应同样具有古怪的计量数):
\begin{center}
    \ce{34Tl2O3 + 24NH3 -> 2Tl[Tl(N3)4] + 30TlOH + 21H2O}
\end{center}
\subsection{铊的氧化物与氢氧化物}
\subsubsection{一价铊的氧化物与氢氧化物}
隔绝空气加热\ce{TlOH}或\ce{Tl2CO3}即可得到黑色的\ce{Tl2O}.\ce{Tl2O}具有吸湿性,与水反应能生成\ce{TlOH}.这是一种强碱.\\
\indent \ce{Tl2O}也可以熔于\ce{B2O3}中形成\ce{TlBO2}.此外,通过共热氧化物也可以制得\ce{TlAlO2}和\ce{TlGaO2}等.
\subsubsection{三价铊的氧化物}
制备\ce{Tl2O3}的方法有很多种.一般而言,用\ce{Cl2}或\ce{Br2}氧化\ce{TlNO3}溶液后将沉淀得到的水合氧化物\ce{Tl2O3.$\dfrac32$H2O}脱水即可得到\ce{Tl2O3}.这是一种棕黑色的具有氧化性的物质.\\
\indent \ce{Tl(OH)3}似乎是不存在的(也有很多观点认为其存在),只有对应比例的水合氧化物.
\subsubsection{混合价态的铊氧化物}
将\ce{Tl2CO3}和\ce{Tl2O3}按照$3:1$的比例混合加热即可得到混合氧化物\ce{Tl4O3}(即\ce{3Tl2O.Tl2O3}).\\
\indent 用\ce{Pt}电极电解\ce{Tl2SO4}和\ce{H2C2O4}的混合溶液即可得到棕紫色的\ce{TlO2}.这显然应当是一种过氧化物,我们可以推测其真实的组成为\ce{[Tl+][Tl^3+][O2^2-]2}.
\subsection{铊的毒性}
\subsubsection{铊的毒理学性质与解毒方法}
\ce{Tl^I}有剧毒.铊对哺乳动物的毒性高于铅,汞等金属元素,与砷相当,其对成人的最小致死剂量为$12\text{ mg/kg}$体重.铊中毒的典型症状有毛发脱落,胃肠道反应,神经系统损伤等.铊具有强蓄积性毒性,可以对患者造成永久性损害,包括肌肉萎缩,肝肾的永久性损伤等.\\
\indent 铊的毒性来源于\ce{Tl+}与\ce{K+}的相似性.细胞膜无法分辨这两种离子,因此人体会主动吸收\ce{Tl+},它们与各种需要\ce{K+}的酶结合,又因为\ce{Tl}与\ce{Ag}的相似性而与各种含硫的蛋白质结合,抑制了各种生理反应,最终造成包括内分泌系统紊乱,神经系统被破坏乃至死亡的严重后果.\\
\indent 对于铊中毒至今没有非常理想的治疗药物.临床上主要使用普鲁士蓝,二巯基丙酸钠,双硫腙等络合剂与\ce{Tl^I}络合后排出体外,同时口服\ce{KCl},利尿剂等加速铊的排出.然而对于中毒造成的损害,至今没有特效的方法.
\subsubsection{历史上的铊中毒案件}
1861年铊被发现后,各种铊盐曾作为鼠药,玻璃添加剂,治疗肺结核等的药物而被广泛使用.然而,\ce{Tl2SO4}等铊盐的无色无味的性质使得其被经常地用于毒药.1961年,Agatha Christie出版了\textit{The Pale Horse}一书,其中的死者们即死于铊中毒.这本书广泛地揭示了铊盐作为毒药的隐蔽性和危害性,也使得其进入了公众的视野.\\
\indent 1994年11月底,就读于清华大学化学系的1992级学生朱令被同系的同学向水杯等常用个人物品中投毒,出现了典型的铊中毒症状\footnote{关于本案的详细信息,可以参见https://zh.wikipedia.org/wiki/朱令铊中毒事件.}.然而,协和医院在收治并检查后明确向家属表示可以排除铊中毒的现象(事实上院方只做了砷中毒的筛查而并没有做铊中毒的筛查),直到症状持续恶化后,朱令的高中同学贝志城,蔡全清等人将她的症状和诊疗措施译成英文后在互联网上求救,才得到了美国加州大学洛杉矶分校的中国留学生李新博士等人的帮助.贝志城和蔡全清等在发出求救电子邮件后,一共收到来自18个国家和地区的2000多份回复.发出求救邮件的18天里,有84位医学专家提出“铊中毒”的诊断.\\
\indent 然而,1995年4月18日,贝志城拿着翻译好的电子邮件到协和医院重症监护区门口给医生参考,但他没有得到积极回应,很少人参看,也没有采纳电子邮件中的铊中毒判断和相应的检测办法,使得当时网上远程诊断的结果没有及时发挥相应的作用.贝志城在回忆此事时写道“协和当年的ICU主任,是他拒绝进行进一步的重金属中毒检查,甚至在发现协和误诊之后,毫无内疚之感,居然在医院会议上说:‘这件事是西方反华势力企图利用此事搞臭中国医疗界’”.直到同年的5月,朱令才被用正确的药物,即普鲁士蓝的治疗.\\
\indent 尽管在得到正确的治疗后,朱令于1995年8月31日从昏迷中苏醒,但由于治疗不及时,昏迷时间太长,留下了视神经萎缩,双下肢瘫,肌萎缩,智能障碍等严重的后遗症,生活无法自理.2023年4月,她被确诊脑瘤,由于脑瘤可能压迫主神经,医生认为她活不过一年.同年11月18日,朱令的脑瘤发作,颅压升高,开始发高烧.同年12月22日22时59分,朱令逝世,终年50岁.\\
\indent 令人难以释怀的是,本案最大的嫌疑人,朱令的室友孙维却并没有受到很严格的调查,现今已经旅居国外.抛开凶手和作案动机不看,朱令案无疑也反映了清华大学,乃至中国各高校化学系对危险化学品的管控混乱,以及协和医院对典型重金属中毒症状诊断以及后续治疗的失职.\\
\indent 两年后的1997年,北京大学又发生了一起铊中毒案件,即化学系94级学生王小龙对同班同学江林及室友陆晨光实施的投毒案.王小龙此前与江林存在数次因为学业等原因的争吵冲突.他心生报复的念头,从实验室先后数次盗窃\ce{Tl2SO4}粉末,加入江林的水杯中,又因为第一次没有起效而对无辜的室友陆晨光痛下毒手,加大了剂量投入陆的奶粉等物品中.5月16日,陆晨光开始有中毒反应,频繁呕吐,很快被送往解放军301医院.5月17日,在江林出现明显症状后,王小龙因害怕两人死亡而自首其投毒的事实.由于诊断和救治及时,两人的身体并没有受到太大损害.1997年,犯罪嫌疑人王小龙经法院审理后,被判故意杀人罪,有期徒刑11年.\\
\indent 即使是在最近几年,仍然有铊中毒的事件发生.2020年12月,游族网络公司董事长暨总经理,实际控制人,控股股东林奇因与下属高管许垚纠纷,被在茶叶中下铊毒.林奇于同年12月25日抢救无效逝世,
\end{document}