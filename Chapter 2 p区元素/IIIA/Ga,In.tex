\documentclass{ctexart}
\usepackage{EC}
\begin{document}
\section{镓,铟及其化合物}
\subsection{镓,铟的单质}
\subsubsection{镓单质}
\paragraph{镓元素的发现与分布}
镓即D. I. Mendeleev于1870年所预言的\tbf{类铝},1875年由P. E. Lecoq de Boisbaudran用光谱仪发现.此元素是纪念法国(拉丁语为$Gallia$)而得名.\ce{Ga}单质的物理性质和化学性质与Mendeleev预言的惊人地类似,成为元素周期律的有力证据.当de Boisbaudran宣布\ce{Ga}的密度是$4.7\text{ g}\cdot\text{cm}^{-3}$而非预测的$5.9\text{ g}\cdot\text{cm}^{-3}$时,Mendeleev写信建议他重新测定\ce{Ga}的密度.结果我们知道,\ce{Ga}单质的密度事实上为$5.904\text{ g}\cdot\text{cm}^{-3}$.Mendeleev的元素周期律获得了巨大的胜利.\\
\indent 尽管镓的丰度几乎是硼的两倍,但其分散的分布使得提炼变得困难.通常,它与相邻的\ce{Zn,Ge}或同族的\ce{Al}共生在一起,以前从焙烧闪锌矿\ce{ZnS}等硫化物矿的烟道灰中收集,但现在是作为铝工业的副产品而制得.中国巨大的铝工业规模使得中国的镓单质产量居世界第一,占世界总产量的$80\%$以上.正因如此,面对以美国为首的西方国家对中国半导体产业的封锁,中国才能实施对镓锗的出口管制这一有力的反制措施\footnote{2023年7月3日,中国商务部与海关总署发布公告,自8月1日起对镓,锗相关物项实施出口管制.}\footnote{巧合的是,同年9月3日举行的第37届中国化学奥林匹克(初赛)的第一题就考察了镓的化合物的结构与性质.}.
\paragraph{镓单质的生产}
在拜耳法中将铝土矿加工成氧化铝的过程中,镓会在氢氧化钠溶液中积累,这些镓可以通过多种方法提取,例如电解精炼法或离子交换树脂分离法.总之,这一过程将得到粗镓单质.如果需要更高纯度的镓用于半导体工业,那么则需要区域熔融技术提纯.
\paragraph{镓单质的物理性质}
在标准状况下,\ce{Ga}单质是质地柔软的银蓝色金属,在液态下则为银白色.\\
\indent \ce{Ga}单质具有相当低的熔点,仅为$29.8\tc$,这意味着它可以放在你的手中而熔化\footnote{尽管\ce{Ga}单质是无毒的,但此操作仍具有危险性,请勿擅自尝试.}.\ce{Ga}单质在凝固时会发生体积膨胀,因此最好不要储存在刚性容器中以免发生破裂.值得注意的是,它是具有最大液态范围的金属之一.与\ce{Hg}不同,\ce{Ga}在高温下具有低蒸气压.\ce{Ga}的沸点为$2673\K$,比它在绝对标度上的熔点高8倍以上,这是任何元素的熔点和沸点之间的最大比值.因此,\ce{Ga}单质可以被用于制造高温温度计.\\
\indent \ce{Ga}会扩散到金属的晶格脆化大部分金属,如扩散到铝锌合金和钢的边界里,使它们变得很脆.\ce{Ga}很容易和其它金属形成合金.液态\ce{Ga}会浸润玻璃,皮肤以及很多材料,使得它更难处理.涂在玻璃上的镓会形成一面明亮的镓镜.
\paragraph{镓单质的结构}
与很多金属不同,\ce{Ga}单质具有非常明显的各向异性.在这一正交晶系的单质中,最近的\ce{Ga-Ga}距离为$244\text{ pm}$,而次近的\ce{Ga-Ga}距离则为$270\text{ pm}$.这表明晶体中存在配对的\ce{\{Ga2\}}单元,它们交错排列形成了\ce{Ga}单质.
\chemfig{Ga}{0.1}{\ce{Ga}单质的晶胞示意图}
\paragraph{镓单质的化学性质与反应}
同\ce{Al}一样,\ce{Ga}单质也是两性的:
\begin{center}
    \ce{2Ga + 6HCl -> 2GaCl3 + 3H2}\\
    \ce{2Ga + 2NaOH + 6H2O -> 2Na[Ga(OH)4] + 3H2}
\end{center}
镓酸盐溶液相比铝酸盐溶液更加容易形成.
\subsubsection{铟单质}
铟也是通过光谱仪发现的.1863年,F. Reich和H. T. Richter首次鉴定了\ce{In}的存在,命名取自其火焰光谱中耀眼的靛蓝色谱线(拉丁语$indicum$).铟是亲硫元素,倾向于和大小相近的Zn结合在一起存在于闪锌矿中.目前工业上是从焙烧\ce{Zn},\ce{Pb}的硫化物矿时所逸出的烟道灰中回收铟.\\
\indent \ce{In}单质是柔软的\footnote{铟是除去碱金属之外最柔软的金属.}具有耀眼光泽的银白色金属,弯曲时由于孪晶的破裂发出高声调的尖叫.\ce{In}的熔点也不高,仅为$156.6\tc$.\ce{In}单质属于四方晶系,点阵形式为体心四方点阵.\\
\indent \ce{In}主要运用在半导体产业,低熔点金属合金如焊料.软金属高真空密封垫以及制造涂覆在玻璃上的透明氧化铟锡(ITO)导电膜.此外,由于\ce{In}的中子吸收截面相当高,因此也被用作核反应堆的控制棒.\\
\indent \ce{In}单质不同于\ce{Al}和\ce{Ga},它不溶于碱中,而只能溶于酸中.
\subsection{镓,铟的氢化物及有关的配合物}
\ce{GaH3}是粘稠的液体,熔点为$-15\tc$,在室温下定量地分解为\ce{Ga}单质和\ce{H2}.\ce{GaH3}可以由以下方法制备:
\begin{center}
    \ce{4LiH + GaCl3 -> LiGaH4 + 3LiCl}\\
    \ce{LiGaH4 + Me3NHCl ->T[\ce{Et2O}] Me3NGaH3 + LiCl + H2}\\
    \ce{Me3NGaH3(s) + BF3(g) -> GaH3(l) + Me3NBF3(s)}
\end{center}
加合物\ce{Me3NGaH3}是无色的晶状化合物,熔点为$70.5\tc$.与\ce{Me3NAlH3}相同,它也能再与一分子\ce{Me3N}反应生成三角双锥构型的\ce{(Me3N)2GaH3}.\\
\indent \ce{GaCl3}与\ce{LiBH4}能反应得到\ce{GaH(BH4)2}:
\begin{center}
    \ce{2GaCl3 + 6LiBH4 -> 2GaH(BH4)2 + 6LiCl + B2H6}
\end{center}
与\ce{Al2B4H18}不同的是,\ce{GaH(BH4)2}并没有二聚,而是采取了与\ce{MeAl(BH4)2}类似的四方锥结构.
\chemfig{GaH(BH4)2}{1}{\ce{GaH(BH4)2}的结构}
而\ce{In}的氢化物\ce{InH3}似乎并不稳定,只有在配体对其配位时才能存在.
\subsection{镓,铟的卤化物及有关的配合物}
\subsubsection{镓,铟的三卤化物}
\paragraph{\ce{GaF3}与\ce{InF3}}
和\ce{AlF3}类似,\ce{GaF3}和\ce{InF3}具有明显较高的熔点,沸点和较大的生成焓.它们最好由\ce{(NH4)3MF6}的热分解制备:
\begin{center}
    \ce{(NH4)3MF6 ->T[$\Delta$] 3HF + 3NH3 + MF3}\ \ \ $\left(\ce{M}=\ce{Ga},\ce{In}\right)$
\end{center}
\ce{GaF3}和\ce{InF3}具有与\ce{AlF3}类似的结构,其中的\ce{Ga}或\ce{In}都是六配位.
\paragraph{\ce{GaCl3}与\ce{InCl3}}
\ce{GaCl3}和\ce{InCl3}与\ce{AlCl3}具有相似的性质.例如,它们都形成二聚分子\ce{M2Cl6},都可以作为好的Lewis酸接受Lewis碱的配位.\\
\indent 同样地,\ce{[GaX4]-}和\ce{[InCl4]-}等四面体形的离子都是存在的,在醚溶液中就可以观测到.然而,由于\ce{H2O}的配位,后者在水溶液中形成\ce{[InCl4(H2O)2]-}等离子,使得其失去$T_{\text d}$对称性.\\
\indent \ce{In^{III}}还能形成\ce{[InCl5]^2-}离子.在\ce{[NEt4]2[InCl5]}中,这一离子罕见的采取$C_{4\text v}$的四方锥构型.这和类似的主族元素的五配位化合物所常采取的三角双锥构型是不同的.应当说明的是,高周期主族元素更加容易出现此现象.三角双锥和四方锥在能量上没有很大差别,配体的位阻,电荷效应,以及晶体中的其它离子都能对具体采取的构型有影响\footnote{事实上,在\ce{[PPh2Cl2]2[InCl5]}里,\ce{[InCl5]^2-}又变回了三角双锥构型.}.
\paragraph{其它三卤化物}
显然可以预见的是,\ce{Ga},\ce{In}的三溴化物和三碘化物相比前面的\ce{MF3}和\ce{MBr3}没有那么稳定.尽管如此,它们也是存在的,并且性质与\ce{MCl3}相似.
\subsubsection{镓,铟的低卤化物}
\paragraph{一卤化物}
\subparagraph{一卤化物的制备}
通常,采取\ce{Ga}还原\ce{GaX3}的方式即可制得\ce{GaX}:
\begin{center}
    \ce{GaX3 + 2Ga -> 3GaX}
\end{center}
当然,由于\ce{GaX}的不稳定性,制得的\ce{GaX}通常都是不纯的.另一种方法是将过量的\ce{Ga}单质与卤素单质(除了氟单质之外)反应,不过这似乎与前面的方法没有本质上的区别.\\
\indent 对于\ce{In}而言,除了\ce{InF}应采取\ce{In}与\ce{InF3}的反应得到外,其它\ce{InX}可以通过\ce{Hg2X2}与\ce{In}单质的反应得到:
\begin{center}
    \ce{2In + Hg2X2 -> 2InX + 2Hg}\ \ \ $\ce{X}=\ce{Cl},\ce{Br},\ce{I}$
\end{center}
\subparagraph{\ce{GaF}和\ce{InF}}
同\ce{AlF}一样,这两种物质是不稳定的气体,容易发生歧化.这也许是因为\ce{MF3}的晶格能相当大以至于非常稳定的缘故.
\subparagraph{GaI}
有必要把这个神奇的物质单独拿出来讨论,不仅是因为第37届中国化学奥林匹克(初赛)的传奇第一题所创飞的众多选手,也是因为它相比\ce{GaF}等类似物要稳定一些,也表现出了奇特的性质.\\
\indent 将液态\ce{Ga}与\ce{I2}在甲苯中超声处理,即可得到淡绿色的不溶性粉末\ce{GaI}.这种成分复杂的物质具有两种可能的结构,一种是\ce{[Ga2][Ga]+[GaI4]-},另一种是\ce{[Ga2][Ga]^+_2[Ga2I6]^2-}.考虑到\ce{Ga}单质中存在的\ce{\{Ga2\}}单元,提出这种结构是相当合理的.这两种阴离子的结构分别如下:
\bichemfig{GaI4-}{1}{\ce{[GaI4]-}的结构}{Ga2I62-}{1}{\ce{[Ga2I6]^2-}的结构}{\ce{GaI}中可能存在的阴离子}
\subparagraph{\ce{GaCl}和\ce{GaBr}}
似乎没什么可说的.它们的稳定性介于\ce{GaF}和\ce{GaI}之间.通常,它们在溶液中是亚稳态的,一旦温度稍高就会发生歧化.\\
\indent \ce{GaCl},\ce{GaBr}和\ce{GaI}经常被用于制备\ce{Ga}的团簇和金属有机化合物.
\subparagraph{\ce{In}的低卤化物}
\ce{In}的低卤化物的性质应当与\ce{Ga}的类似物相似.
\paragraph{二卤化物}
\subparagraph{二卤化物的制备}
将\ce{Ga}单质与等量的\ce{GaX3}(其中$\ce{X}=\ce{Cl},\ce{Br},\ce{I}$)混合加热即可得到\ce{GaX2}:
\begin{center}
    \ce{GaX3 + Ga -> 2GaX2}
\end{center}
同样的方法也可以制得\ce{InCl2}.\\
\indent 当然,用化学计量的\ce{Hg2X2}与\ce{Ga}单质反应亦可得到\ce{GaX2}.
\begin{center}
    \ce{Ga + Hg2X2 -> GaX2 + 2Hg}
\end{center}
\subparagraph{二卤化物的性质}
事实上,\ce{MX2}\footnote{这里$\ce{M}=\ce{Ga},\ce{In}$,$\ce{X}=\ce{Cl},\ce{Br},\ce{I}$.在不引起歧义的情况下,下同.}是成分为\ce{M^{I}[M^{III}X4]}的稳定的离子化合物.像\ce{B2X4}那样形成\ce{M-M}键对这些元素毕竟还是太弱,不如采取混合价态.将\ce{MX}与\ce{AlX3}等Lewis酸反应也可以得到类似成分的\ce{[M]+[AlX4]-}等离子.\ce{GaX2}本身也可以和各种配体反应生成\ce{[GaL4]+[GaX4]-}这样的物质.
\paragraph{低卤化物的衍生物}
在各种\ce{MX2}的衍生物中,就普遍地存在含\ce{M-M}键的\ce{G2X4}结构单元了.例如,\ce{Ga2Br4py2}的结构如下所示:

\end{document}