\documentclass[draft]{ctexart}
\usepackage{EC}
\begin{document}
\section{碳及其化合物}
\subsection{单质碳}
\subsubsection{石墨}
\paragraph{石墨的结构,性质与反应}
众所周知,石墨具有六边形\ce{\{C6\}}环稠合的二维层状结构.
\chemfig{graphite-1}{1}{石墨的二维层状结构示意图}
根据层间重叠方式的不同,石墨有两种异构体,即六方$\alpha$-石墨和三方$\beta$-石墨,前者在自然界中占据主要(约$70\%$).两者的晶体结构如下所示.
\bichemfig{alpha-graphite}{1}{六方$\alpha$-石墨的晶体结构示意图}{beta-graphite}{1}{三方$\beta$-石墨的晶体结构示意图}{石墨的两种异构体(黑点表示碳原子,虚线表示晶胞边界)}
从密堆积的角度描述这两种石墨的结构,六方$\alpha$-石墨的堆积层可以表示为$\cdots AB\overline{AB}AB\cdots$,而三方$\beta$-石墨的堆积层可以表示为$\cdots ABC\overline{ABC}ABC\cdots$.研磨$\alpha$-石墨可以使其转化为$\beta$-石墨,而加热$\beta$-石墨可以使其转化为$\alpha$-石墨.不完全的转化将得到湍层石墨,其中各层的堆积次序是随机的.\\
\indent 由于每一层均可以视作无限稠合的芳环,因而电子在层内是高度离域的,这使得石墨具有良好的导电性能,并且平行于层方向的电导率大约是垂直于层方向的$10^3$倍.\\
\indent 石墨的层间间距较大,只有不强的范德华力作用,使得其容易沿层平面滑动解理\footnote{\tbf{解理}(cleavage)又称\tbf{劈理},是矿物学和宝石学的常见术语,指的是矿物或宝石晶体在外力的作用下,沿一定的结晶学方向裂开成光滑平面的性质.这些光滑平面被称为\tbf{解理面}或\tbf{劈理面}.}.因此,石墨的硬度很低.\\
\indent 石墨的独特的层状结构使得它与\ce{F2}反应时控制条件就能形成理想化学式为\ce{CF}的分子.这种分子由无限延伸的椅式六元环稠合而成,所有\ce{F}原子分居平面两侧.
\paragraph{石墨的分布,生产与用途}
晶状的石墨广泛分布于全世界,然而大多数以薄片的形式存在于硅酸盐矿石中,几乎没有经济价值.由于缺乏天然的来源,石墨在工业上主要通过焦炭与二氧化硅的反应制备:
\begin{center}
    \ce{SiO2 + 3C -> SiC}\\
    \ce{SiC(s) ->T[$2500\tc$] Si(g) + C(s,\text{石墨})}
\end{center}
石墨的主要用途如下:
\begin{enumerate}[label=\tbf{\arabic*.},topsep=0pt,parsep=0pt,itemsep=0pt,partopsep=0pt]
    \item \tbf{作为铅芯的主要成分制造铅笔}\\
        石墨的层状结构易剥落,可用于纸上书写.现代铅笔笔芯以石墨和黏土制造,石墨含量越高,铅芯越软,颜色越黑.\\
        铅笔之名源自其早期雏形为铅金属所制造,而后则又因欧洲中世纪时石墨被误以为是铅的一种,而有了\tbf{黑铅}这个名称,因此铅笔一词在各语言中流传使用而未修正.中国古代的铅笔事实上确为铅粉所制,并没有石墨.
    \item \tbf{惰性电极材料}\\
        石墨的优良的导电性和稳定性使得其可以作为电极材料.这一方案显然比铂电极更加便宜,并且不会受到卤素的侵蚀.因此,电解\ce{NaCl}水溶液制备\ce{Cl2}和\ce{NaOH}时通常采用石墨电极.
    \item \tbf{核反应堆的中子减速剂}
        石墨可以有效地减缓核反应堆里中子的速度,有效控制反应进行.石墨耐高温,纯度高,是迄今为止核反应堆中极为重要的原材料.
\end{enumerate}
\subsubsection{金刚石与蓝丝黛尔石}
\paragraph{金刚石的结构与性质}
在金刚石中,\ce{C}原子以$\text{sp}^3$杂化形式与相邻的\ce{C}原子成键,从而形成无限延伸的三维结构.金刚石也有两种异构体.大多数天然得到的或人工合成的金刚石均属立方晶系,因此一般而言我们所称的金刚石为立方金刚石.1967年在美国亚利桑那州的巴林杰陨石坑第一次鉴别出了六方金刚石,并以爱尔兰晶体学家D. K. Lonsdale\footnote{1929年, D. K. Lonsdale首次利用X射线衍射法证明了苯是平面的.1931年,她又首次利用Fourier变换光谱分析法解析了六氯苯的结构,为推动芳香性的研究做出了重大贡献.她是英国皇家化学会首次选入的两个女会士之一,伦敦学院大学的首位终身女教授,国际晶体学会第一位女主席,英国科学促迸学会首位女主席.}命名为蓝丝黛尔石.两者的结构如下所示.
\bichemfig{diamond-1}{0.1}{立方金刚石的晶体结构}{diamond-2}{0.1}{六方金刚石的晶体结构}{两种金刚石的晶体结构}
事实上,分别将闪锌矿和纤锌矿中的所有原子替换为碳原子即可分别得到立方金刚石和六方金刚石.\\
\indent 蓝丝黛尔石推测为流星上的石墨坠入地球时在高温高压下形成.它保留了石墨的平行六边形结构,但层间的碳原子成键.蓝丝黛尔石相较立方金刚石不稳定,这可能是因为其中存在船式六元环(即存在重叠式构象的碳原子)所致.有研究表明,蓝丝黛尔石具有比立方金刚石更高的硬度和更强的抗压能力.然而,天然存在的蓝丝黛尔石不纯,且并不完全为六方结构.\\
\indent 金刚石是已知硬度最高的天然物质.同时,它的化学性质也非常稳定,即使在纯\ce{O2}中也要加热到$720\tc$才能燃烧.
\paragraph{金刚石的合成与应用}
由于金刚石的硬度极高且导热性极高,因此常用于用于钻探,研磨工具之上,可以用来切削和刻划其他物质.自1955年通用电气发明通过高温高压处理石墨获得人造金刚石的技术以来,人们已经可以利用气相沉积法等制成金刚石微粒,因此现在细小颗粒的人工合成钻石已经较同级天然钻石便宜.故此,天然钻石的工业价值已经完全消失,目前的主要用途已仅限于首饰与观赏.随着人工合成技术的成熟,合成钻石也进入了首饰市场,但总是遭到天然钻石公司的诋毁.
\subsubsection{石墨烯}
\paragraph{石墨烯的结构与性质}
石墨烯又称单层石墨,碳单层,是由石墨剥层制造出的碳单质.石墨烯的碳原子以$\text{sp}2$杂化轨道组成六角型呈蜂巢晶格的单层平面薄膜,其厚度仅相当于1个碳原子的直径.\\
\indent 石墨烯几乎是完全透明的,它的透光率为$97.7\%$.与石墨类似的,它具有极高的导热和导电性能.
\paragraph{石墨烯的制备与应用}
2004年,曼彻斯特大学和俄国切尔诺戈洛夫卡微电子工艺研究所的两组物理团队共同合作,首先分离出单独石墨烯平面.海姆和团队成员偶然地发现了一种简单易行的制备石墨烯的新方法,他们将石墨片放置在塑料胶带中,折叠胶带粘住石墨薄片的两侧.撕开胶带,薄片也随之一分为二.不断重复这一过程,就可以得到越来越薄的石墨薄片,而其中部分样品仅由一层碳原子构成,这就制得了石墨烯.\\
\indent 除此之外,还可以采用在\ce{Ni}基底上进行气相沉积法等方法.总之,由于石墨烯的广泛的潜在用途,人们仍然在对改进工艺,提高制备效率和纯度等方面进行努力.\\
\indent 石墨烯凭借其超凡的导电性,导热性,机械强度,柔韧性和透明度,在众多领域展现出革命性的应用潜力.在电子领域,它是制造超快芯片,柔性透明电极(用于可折叠屏幕和触摸屏)的理想材料;在能源领域,它能显著提升锂电池,超级电容器的充放电速度和容量,并用于高效太阳能电池;在材料领域,它用于制造更轻,更强,更耐用的复合材料和防腐,导电,导热涂层;此外,它在高灵敏度传感器,高效海水淡化膜,水处理过滤膜,生物医学(如药物递送,生物传感)以及可穿戴技术等方面也具有广阔前景,被视为推动未来科技发展的关键材料之一.
\subsubsection{石墨炔}
\subsubsection{富勒烯及其衍生物}
\paragraph{富勒烯的制备}
富勒烯又称球碳,是由纯碳原子组成的球形分子.这类分子可以通过在石墨电极间放电后形成的碳烟中分离得到,也可以在严格控制下使苯不完全燃烧,在形成的碳烟中分离得到.
\paragraph{足球烯的结构与反应}
1985年英国化学家哈罗德·沃特尔·克罗托博士和美国科学家理查德·斯莫利在莱斯大学制备出了第一种富勒烯,即\tbf{足球烯}\ce{C_60}.它的骨架结构与足球一致,属于$I_{\text h}$点群,具有很高的对称性,同时也是最常见和最稳定的富勒烯.
\begin{figure}[H]
    \centering
    \subfigure[\ce{C60}的结构]{
        \centering\begin{minipage}[b]{.3\linewidth}
            \includegraphics[scale=0.75]{picture/C60-1.eps}
        \end{minipage}
    }
    \subfigure[\ce{C60}的成键方式示意图]{
        \centering\begin{minipage}[b]{.3\linewidth}
            \includegraphics[scale=0.75]{picture/C60-2.eps}
        \end{minipage}
    }
    \subfigure[\ce{C60}的平面示意图]{
        \centering\begin{minipage}[b]{.3\linewidth}
            \includegraphics[scale=0.75]{picture/C60-3.eps}
        \end{minipage}
    }\vspace{-10pt}\caption{\ce{C60}分子的结构}
\end{figure}
\ce{C60}一共有12个五边形面和20个六边形面,其中每个五元环都与五个六元环相接,而每个六元环都与三个五元环和三个六元环交替相接.不难看出,\ce{C60}中的所有\ce{C}原子均等价.
\paragraph{\ce{C60}的体积}
\ce{C60}中两个六元环共用\ce{C-C}键的键长为$138.8\text{ pm}$,五元环和六元环共用\ce{C-C}键的键长为$143.2\text{ pm}$.根据键长数据,我们可以求出\ce{C60}分子的体积\footnote{当然,我们可以用内切球,外接球等方法近似计算其体积.这里采取精确计算的办法.}.\\
\indent 不难看出,\ce{C60}的结构与三角二十面体有很大的相似性.为此,我们可以把所有两个六元环共用的边延长,将\ce{C60}补全为一个正三角二十面体.这一过程的逆过程即在三角二十面体的每个顶点截去一个正五棱锥.由于原来的三角二十面体的每个面都是正三角形,因此截面的五边形的边长与五棱锥的棱长相等.因此,这个正三角二十面体的边长为
\[a=d_{66}+2d_{56}=(138.8+143.3\times2)\text{pm}=425.2\text{ pm}\]
经过三角二十面体的对棱(不妨记为$A_1A_2$和$A_3A_4$)取截面,那么$A_1A_2A_3A_4$应当是一个矩形,矩形的中心$O$即为三角二十面体的中心.短边长度$d_1=a=425.2\text{ pm}$,长边即五边形对角线,其长度
\[d_2=\sqrt{2a^2\left(1-\cos108^\circ\right)}=688.0\text{ pm}\]
于是中心$O$到顶点$A$的距离为
\[r=\dfrac{\sqrt{d_1^2+d_2^2}}{2}=404.4\text{ pm}\]
取任意三角形面的面心$B$,中心$O$到$B$的距离为
\[h_1=\sqrt{r^2-\left(\dfrac{\sqrt3}{3}a\right)^2}=321.3\text{ pm}\]
于是该三角二十面体的体积为
\[V_1=20\cdot\dfrac13\cdot\dfrac{\sqrt3}{4}a^2h_1=1.677\times10^{8}\text{ pm}^3\]
\indent 现在我们计算截去的五棱锥的体积.五棱锥的高为
\[h_2=\sqrt{d_{56}^2-\left(\dfrac{d_{56}}{2\cos54^\circ}\right)^2}=75.28\text{ pm}\]
于是五棱锥的体积为
\[V_2=\dfrac13\cdot\left(5\cdot\dfrac12\cdot d_{56}\cdot\dfrac{\tan54^\circ d_{56}}{2}\right)\cdot h_2=8.854\times10^5\text{ pm}^3\]
\indent 对于\ce{C60}而言,一共截去了$12$个五棱锥,因此\ce{C60}的体积为
\[V=V_1-12V_2=1.571\times10^{8}\text{ pm}^3\]
\indent 这是一道锻炼空间想象和立体几何能力的题目.你还可以采取其它计算方法得到它的体积.
\paragraph{\ce{C60}中碳的杂化形式}
在探讨这一问题之前,我们需要了解一些关于杂化的知识.$\text{sp}^i$杂化的轨道的波函数为
\[\psi_{\text{sp}^i}=a\phi_{\text s}+b_x\phi_{\text{p}_x}+b_y\phi_{\text{p}_y}+b_z\phi_{\text{p}_z}\]
其中$\phi_{\text s}$和$\phi_{\text{p}_x},\phi_{\text{p}_y},\phi_{\text{p}_z}$分别为$s$轨道和三个$p$轨道的波函数.为了方便考虑,我们最好让$\text{p}$轨道归一化.令
\[b=\sqrt{b_x^2+b_y^2+b_z^2}\ \ \ \ \ \phi_{p}=\dfrac{b_x\phi_{\text{p}_x}+b_y\phi_{\text{p}_y}+b_z\phi_{\text{p}_z}}{\sqrt{b_x^2+b_y^2+b_z^2}}\]
你可以将$\phi_{\text{p}_x},\phi_{\text{p}_y},\phi_{\text{p}_z}$分别视作空间中三个方向上的单位向量,上述过程实际上是求出三个$\text p$轨道线性组合后指向方向的单位向量,这就是归一化$\text p$轨道的意义.由于$\text{s}$轨道并无特殊取向,因此$\text{sp}^i$杂化的轨道取向由$\phi_p$的取向决定.\\
\indent 现在再来考虑组合后的波函数本身应当满足的性质:归一性和正交性.为了方便,下面就用向量的形式表示函数.对于轨道$\phi_{\text{sp}^i}$,归一性要求
\[a^2+b^2=1\]
现在考虑两个$\phi_{\text{sp}^i}$轨道,将它们分别表示为
\[\overrightarrow{\psi_{\text{sp}^i_A}}=a\overrightarrow{\phi_{\text{s}}}+b\overrightarrow{\phi_{\text{p}_A}}\ \ \ \ \ \overrightarrow{\psi_{\text{sp}^i_B}}=a\overrightarrow{\phi_{\text{s}}}+b\overrightarrow{\phi_{\text{p}_B}}\]
它们之间的夹角,即键角为$\theta$,这意味着决定两者方向的$p$轨道函数的夹角为$\theta$.正交性要求
\[\overrightarrow{\psi_{\text{sp}^i_A}}\cdot\overrightarrow{\psi_{\text{sp}^i_B}}=0\]
由于$s$轨道和任意方向上的$p$轨道都是正交的,于是$\overrightarrow{\phi_{\text{s}}}\cdot\overrightarrow{\phi_{\text{p}}}=0$.于是上式即为
\[a^2+b^2\overrightarrow{\phi_{\text{p}_A}}\cdot\overrightarrow{\phi_{\text{p}_B}}=0\]
我们说过,$\overrightarrow{\phi_{\text{p}}}$的几何意义为指向空间中某一方向的单位向量,那么两个夹角为$\theta$的单位向量的内积显然应当是$\cos\theta$.于是就有
\[a^2+b^2\cos\theta=0\]
将归一化条件代入其中就有
\[a^2=-\dfrac{\cos\theta}{1-\cos\theta}\ \ \ \ \ b^2=\dfrac{1}{1-\cos\theta}\]
采取$\text{sp}^i$意味着$\text{s}$轨道和$\text{p}$轨道的比例为$1:i$,这可以用线性组合系数的平方之比来表示,这样就有
\[i=\dfrac{b^2}{a^2}\]
结合上面的式子就有
\[1+i\cos\theta=0\]
或者
\[\theta=\arccos\left(-\dfrac1i\right)\]
这样,你就可以计算各种$\text{sp}^i$杂化中$i$的具体数值,例如\ce{H2O}中的\ce{O}接近$\text{sp}^4$杂化.\\
\indent 对于\ce{C60}而言,我们可以近似地采取平均键角进行计算.每一个碳都被两个六边形和一个五边形所共用,平均键角
\[\overline{\theta}=\dfrac{108^\circ+2\times120^\circ}{3}=116^\circ\]
于是杂化指数
\[i=-\dfrac{1}{\cos116^\circ}=2.28\]
这就是\ce{C60}中的\ce{C}原子采取$\text{sp}^{2.28}$杂化的由来.
\paragraph{\ce{C60}的晶体结构}
正常情况下,\ce{C60}晶体属立方晶系,\ce{C60}分子按照立方最密堆积形成晶体.\\
\indent 说到这一晶体的结构,就不得不提到\ce{K3C60}这一化合物.其中\ce{K+}填入所有\ce{C60}形成的四面体空隙和八面体空隙.这一物质的有趣的一点是其中\ce{K}的质量分数为$14.000\%$\footnote{所谓蕉下客.}.
\paragraph{\ce{C60}的化学性质}
尽管结构中大量存在苯环,\ce{C60}的芳香性却并不像苯一样如此明显.从键长数据也可以看出,\ce{C60}中的双键更多时候趋向于定域化,这意味着它能发生与烯烃类似的反应.此外,大的共轭体系也使得其能被单电子氧化/还原.以下是一些反应的例子.
\begin{enumerate}[label=\tbf{\arabic*.},topsep=0pt,parsep=0pt,itemsep=0pt,partopsep=0pt]
    \item \tbf{加成反应}\\
        \ce{C60}能被各种手段加氢.Birch还原能将其还原为\ce{C60H32},用二氢化蒽能将其还原为\ce{C60H36},等等.然而,\ce{C60H60}却因为键角的缘故而难以得到.\\
        \ce{C60}也可以与卤素反应.与\ce{F2}的反应倾向于生成$1,2$位的二取代物,而与\ce{Cl2}和\ce{Br2}的反应倾向于加成到相距较远的\ce{C}原子上,例如\ce{C60Br8}和\ce{C60Br24}.\\
        \ce{C60}也可以与\ce{PhCN2}等卡宾前体反应生成含有三元环的\ce{C61Ph2}.它也可以自己发生$[2+2]$环加成反应二聚成为\ce{C120}.
    \item \tbf{氧化/还原反应}\\
        在合适的氧化剂存在下,可以形成\ce{C60^n+};在合适的还原剂存在下则可以形成\ce{C60^n-}.
\end{enumerate}
\paragraph{其它富勒烯}
除去最常见的\ce{C60}之外,还有\ce{C70},\ce{C84}等富勒烯分子.它们也是笼状结构的分子,但对称性并没有\ce{C60}高.
\subsubsection{碳纳米管}
\subsubsection{无定形碳}
\paragraph{无定形碳的分布}
碳在史前就被认为是一种物质(炭,烟灰),而学界承认碳是一种元素却是18世纪几个实验的结论.大部分无定形碳都含有石墨的层状结构,然而晶体化程度很低,没有固定形状和周期性结构规律,同时经常掺杂很多其它元素.
\paragraph{焦炭}
煤经过高温碳化产生的焦炭是一种石墨化程度很差的碳,其大多数用于高炉炼钢.
\paragraph{碳黑}
碳黑是由液体烃或天然气的不完全燃烧制得的.这种颗粒很小的碳单质主要用于橡胶工业以强化橡胶,也可以用于墨水,油漆,纸张和塑料中的颜料.
\paragraph{活性炭}
活性炭是是黑色粉末状或颗粒状的碳物质.活性炭在微观上的不规则排列使得其在交叉连接之间有细孔,在活化时会产生碳组织缺陷,因此它是一种多孔的物质,具有很大的比表面积.\\
\indent 正因如此,活性炭广泛地用于脱色剂(在蔗糖工业中),空气和水体精华以及催化剂载体.它本身也可以催化一些反应,例如\ce{CO}与\ce{Cl2}反应生成\ce{COCl2}等.
\subsection{碳化物}
和硼化物一样,碳化物的种类也十分繁多,且彼此之间似乎也难以有清晰的界限.这里主要分出几类分别进行介绍.
\subsubsection{石墨间充化合物}
石墨中互相平行的碳原子平面层之间的距离比较大,而且层间的主要作用力是较弱的范德华力,这使得大量物质在温和条件下可能插入平面问得到不同组分的片
状化合物.1926年,人们发现石墨与\ce{K}蒸汽在$300\tc$能直接反应形成青铜色的\ce{KC8},这是第一个被制备出的碱金属石墨化物.它的晶体结构示意如下:
\bichemfig{KC8-1}{0.1}{\ce{KC8}的晶体结构}{KC8-2}{0.75}{\ce{KC8}的层投影}{\ce{KC8}的结构}
\ce{KC8}中的石墨层的$c$轴投影相互重叠,两层石墨层间的\ce{K}原子按照上图的方式分布,然后依$\cdots ABCABC\cdots$的方式堆积形成三方晶系的晶体.\\
\indent 除了\ce{KC8}以外,还可以用碱金属制得\ce{MC12},\ce{MC24},\ce{MC36},\ce{MC48}和\ce{MC60}等化合物.它们的结构都以\ce{KC8}为基础,但金属层缺失了图5(b)中心的\ce{M}原子(相当于有规律地缺失了$\dfrac13$的\ce{M}原子),因而形成\ce{MC_{12n}}系列的化合物.如果每层石墨层间都有这样的金属层,即为\ce{MC12};相比\ce{MC12},金属层隔层出现即为\ce{MC24},依此类推.\\
\indent 碱土金属和过渡金属能形成\ce{MC6},其金属层的填充方式也如下所示.
\begin{figure}[H]
    \centering
    \subfigure[\ce{MC12}的结构及其与\ce{MC8}的联系]{
        \begin{minipage}[b]{.55\linewidth}
            \centering\includegraphics{picture/MC12.eps}
        \end{minipage}
    }
    \subfigure[\ce{MC6}的结构]{
        \begin{minipage}[b]{.35\linewidth}
            \centering\includegraphics{picture/MC6.eps}
        \end{minipage}
    }
\end{figure}
这些金属石墨化物有着极高的电导率(大约是石墨的十倍)以及更鲜明的颜色.这意味着它们的本质应当是金属将电子转移至石墨层的导带而形成了二维层状阴离子,阳离子则镶嵌其中.\\
\indent 金属石墨化合物(尤其是碱金属石墨化合物)在空气中极为活泼,遇水可能发生爆炸.\ce{KC8}经常作为各类反应中的强还原剂而使用,尤其是制备各类低价金属配合物时.还原得到的\ce{C}单质以石墨的形式存在,容易处理.
\subsubsection{离子型碳化物}
\paragraph{似盐型碳化物}
某些碳化物中的碳显$-4$价,它们的水解产生\ce{CH4}.典型的此类化合物有\ce{Be2C}和\ce{Al4C3}.两者分别可以由如下的反应得到:
\begin{center}
    \ce{2BeO + 2C ->T[$2000\tc$] Be2C + CO}\\
    \ce{4Al + 3C ->T[高温] Al4C3}
\end{center}
前者具有反萤石型的结构.后者的结构则比较复杂,但其中的\ce{Al}与大多数情况相似,被\ce{C}原子四配位的.
\paragraph{含有\ce{C2}单元的碳化物}
这类碳化物是为人所熟知的,其中最著名的就是炔化钙\ce{CaC2}(通常称作\tbf{电石}).在工业上按照下面的方法制备\ce{CaC2}:
\begin{center}
    \ce{CaO + 3C ->T[$2200\sim2500^\tc$] CaC2 + CO}
\end{center}
在过去,\ce{CaC2}作为\ce{C2H2}的主要来源,现在则更多的来源于石油工业.\ce{CaC2}与水反应即生成乙炔,与\ce{N2}反应生成氰氨化钙,后者的水解生成氨基氰这一化肥工业的重要原料:
\begin{center}
    \ce{CaC2 + 2H2O -> C2H2 + Ca(OH)2}\\
    \ce{CaC2 + N2 ->T[$1000\sim1200^\tc$] CaCN2 + C}\\
    \ce{CaCN2 + 2H2O -> H2NCN + Ca(OH)2}
\end{center}
\ce{CaC2}是无色固体.常温下的\ce{CaC2}中\ce{CA^2+}与\ce{C_2^2-}按照\ce{NaCl}的形式排列,由于\ce{C2^2-}是线形阴离子而变为四方晶系.加热后的\ce{CaC2}则属于立方晶系,对称性的升高源自\ce{C_2^2-}排列方向的随机化,完全转变为\ce{NaCl}型.
\bichemfig{CaC2-1}{0.1}{常温下\ce{CaC2}的晶体结构}{CaC2-2}{0.1}{高温下\ce{CaC2}的晶体结构}{\ce{CaC2}的晶体结构}
与\ce{CaC2}类似的还有镧系元素形成的\ce{LnC2}和\ce{Ln4(C2)3}.与\ce{CaC2}不同,大多数\ce{LnC2}具有良好的导电性,这说明它们之中的\ce{Ln}应当以\ce{Ln^3+}的形式存在,多余的自由电子在\ce{C2^2-}的反键轨道上部分离域,因此其中的\ce{C-C}键的键长相较\ce{CaC2}明显更长.这也使得它们的水解产物比较复杂,经常有\ce{C2H4}等还原产物的生成.
\subsubsection{填隙型碳化物}
填隙碳化物是难溶的,极坚硬的,耐火的材料,并保持了许多金属的特性(例如具有金属光泽和很高的电导率),因此把它们作为合金看待似乎更加合适.这其中最典型的化合物是\tbf{渗碳体}\ce{Fe3C}.渗碳体是钢铁中最硬的相,是钢铁获得高强度,高硬度和耐磨性的基础.它具有复杂的晶体结构,属于正交晶系:
\chemfig{Fe3C}{0.12}{\ce{Fe3C}的晶体结构}
\subsection{碳的氧化物及碳酸盐}
碳形成两种稳定的,也是为人所熟知的氧化物,即\ce{CO}和\ce{CO2}.此外,还有一些不大稳定的氧化物,例如\ce{C3O2}等等.
\subsubsection{碳的复杂氧化物}
\paragraph{二氧化三碳\ce{C3O2}}
\ce{C3O2}的正式名称为1,2-丙二烯-1,3-二酮,是有腐败气味的气体,由丙二酸用\ce{P4O10}脱水得到.正如化学式所描述的那样,这一分子具有\ce{O=C=C=C=O}的线性结构.\ce{C3O2}容易发生聚合,形成六元环内酯并接的链状结构.
\chemfig{C3O2}{1}{\ce{C3O2}聚合后的结构}
\ce{C3O2}容易与\ce{H2O}发生反应再次形成丙二酸,与\ce{HCl}或\ce{NH3}反应则得到对应的酰氯和酰胺,因此可以将它视作丙二酸酐.
\paragraph{苯六酸酐\ce{C12O9}}
苯六酸酐又称蜜石酸酐,是苯六酸(即蜜石酸\footnote{这一名字来源于蜜蜡石\ce{Al2[C6(COO)6].16H2O}.})的酸酐,可以由苯六酸与酰氯或酸酐的反应得到:
\chemfig{C12O9-1}{1}{\ce{C12O9}的制备方式与结构}
从结构与性质上看,把它归为有机物似乎更合适,但它毕竟不含\ce{H}原子,因此放到氧化物一节稍作介绍.\\
关于苯六酸酐的另一个有趣的事实是用浓硫酸氧化石墨和足球烯等部分碳单质都能得到苯六酸酐,这可能是因为这些物质中本身就具有芳环结构.并且两者氧化的理论产率分别为$\dfrac67$和$\dfrac45$.这可以用下面的图来解释.
\bichemfig{C12O9-2}{0.9}{从石墨得到\ce{C12O9}}{C12O9-3}{0.8}{从\ce{C60}得到\ce{C12O9}}{由碳单质得到\ce{C12O9}的原子来源示意图}
\subsubsection{一氧化碳\ce{CO}}
\paragraph{\ce{CO}的结构}
我们都知道\ce{CO}是\ce{N2}的等电子体,键长为$112.8\text{ pm}$,相较普通的羰基(作为对比,\ce{HCHO}中的\ce{C=O}键键长为$120.8\text{ pm}$)长度更短,这意味着其中的键级应当大于$2$.\ce{CO}的极高的键能\footnote{似乎是已知共价键中键能最高的.}$1072\kJm$,以及较羰基明显更大的红外振动波数($2143\text{ cm}^{-1}$)也支持这一观点.理论计算的结果表明,\ce{CO}中的实际键级约为$2.6$.
\chemfig{CO-1}{1}{\ce{CO}的共振式}
尽管\ce{O}的电负性明显地大于\ce{C},但为了满足八电子结构,\ce{O}提供了一对电子额外形成$\pi$键,这也造成了\ce{CO}的偶极矩反常的小,并且事实上(正电荷指向负电荷)由\ce{O}指向\ce{C},这可以由上图右边的共振式的形式电荷反映出来.负电荷端在\ce{C}原子也意味着\ce{CO}在作为配体时几乎都用\ce{C}端作为配位原子.然而,当配位后,由于反馈$\pi$键的形成,负电荷端有时又回到\ce{O}上.\\
\indent \ce{CO}的基态是单线态,没有不成对的电子.
\paragraph{\ce{CO}的物理性质}
\ce{CO}是无色无味的剧毒气体,难溶于水,容易燃烧,有剧毒.
\paragraph{\ce{CO}的制备,化学性质与反应}
\ce{C}在有限量的氧气中或空气中直接氧化成\ce{CO},充足供氧时产生\ce{CO2}.温度较高时,\ce{CO2}倾向于分解生成\ce{CO},因此在高温或还原剂过量的情况下一般的产物均为\ce{CO}.
\subparagraph{\ce{CO}的工业制备与应用}
\ce{CO}以发生炉煤气或水煤气的形式广泛作为燃料使用.发生炉煤气由空气从红热的焦炭上通过得到,主要成分为\ce{CO}和\ce{N2},还含有少部分的\ce{CO2}和痕量的其它气体杂质.发生的反应为:
\begin{center}
    \ce{2C + O2 -> 2CO}\\
    \ce{C + O2 -> CO2}
\end{center}
水煤气则是将水蒸气从红热的焦炭上通过得到,其主要成分为\ce{CO}和\ce{H2}
\begin{center}
    \ce{C + H2O -> CO + H2}
\end{center}
\ce{CO}在工业上主要作为燃料和还原剂使用.关于其还原性,可以参考一般无机化学书上的Ellingham图以获取相关信息.
\subparagraph{\ce{CO}的实验室制备}
\indent 纯净的\ce{CO}可以用浓硫酸对甲酸或草酸脱水得到:
\begin{center}
    \ce{HCOOH ->T[浓硫酸] H2O + CO}\\
    \ce{H2C2O4 ->T[浓硫酸] H2O + CO + CO2}
\end{center}
然而\ce{CO}不能与水反应得到甲酸,因此不将其视作甲酸酐.\\
\indent 另外的方法包括加热锌粉与\ce{CaCO3}的混合物,或硝酸银与碘仿的反应:
\begin{center}
    \ce{Zn + CaCO3 -> ZnO + CaO + CO}\\
    \ce{CHI3 + 3AgNO3 + H2O -> 3HNO3 + CO + AgI}
\end{center}
\subparagraph{\ce{CO}的毒理学}
\ce{CO}可以与血红蛋白结合,结合的能力是\ce{O2}的$300$多倍,因此吸入\ce{CO}将抑制红细胞运输氧,致人失去知觉并死于窒息,这就是\ce{CO}的毒性来源.轻微的中毒可以通过呼吸新鲜空气解除,并且没有后遗症.频繁发生的室内\ce{CO}中毒案件通常是源于老旧的热水器或煤气灶的煤气泄漏,或是室内烧烤时不注意通风导致,并经常造成数人死亡的惨痛后果\footnote{为了您的健康,请使用天然气灶,电热水器,并且不在
再室内用炭炉烧烤.}.人们通常在煤气中添加诸如硫醚等具有明显气味的物质,以提醒使用者注意煤气泄漏.
\subparagraph{\ce{CO}的检测}
\ce{CO}能将水溶液中的\ce{PdCl2}还原成金属\ce{Pd},借由生成的黑色沉淀即可检验\ce{CO}.此法可能受到其它还原性气体的干扰.
\begin{center}
    \ce{CO + PdCl2 + H2O -> CO2 + 2HCl + Pd}
\end{center}
定量地测定\ce{CO}可以通过\ce{CuCl}的酸性溶液吸收形成加合物:
\begin{center}
    \ce{CuCl + CO + 2H2O -> Cu(CO)Cl(H2O)2}
\end{center}
\subparagraph{{\ce{CO}}的部分反应}
\ce{CO}在高温下与碱金属氢氧化物反应形成甲酸盐,与甲氧基化合物反应生成乙酸盐:
\begin{center}
    \ce{NaOH + CO -> HCOONa}\\
    \ce{MeONa + CO -> MeCOONa}
\end{center}

\indent \ce{CO}用碱金属的液氨溶液还原即可得到线性的\ce{C2O2^2-}.\\
\indent \ce{CO}和\ce{Cl2}或\ce{Br2}反应分别得到\ce{COCl2}或\ce{COBr2}.\ce{CO}与液态\ce{S}反应得到\ce{COS}.\\
\indent \ce{CO}与\ce{B2H6}在高压下反应得到对称的加合物\ce{H3BCO},但存在\ce{NaBH4/THF}时则生成环状的\ce{B3O3Me3}:
\begin{center}
    \ce{B2H6 + 2CO -> H3BCO}\\
    \ce{3B2H6 + 6CO -> 2B3O3Me3}
\end{center}
两者的结构如下所示.
\bichemfig{H3BCO}{1}{\ce{H3BCO}的结构}{B3O3Me3}{1}{\ce{B3O3Me3}的结构}{\ce{CO}与\ce{B2H6}的反应产物的结构}
\ce{CO}最重要的性质就是与过渡金属形成羰基化合物.这部分值得单独作为一个章节进行介绍,在这里就暂时略过.
\subsubsection{二氧化碳\ce{CO2}}
\paragraph{\ce{CO2}的物理性质}
\ce{CO2}是无色无味的气体,不可燃也不能助燃.\ce{CO2}可溶于水,但溶解度不大.固态的\ce{CO2}俗称\tbf{干冰},升华点为$-78\tc$\footnote{低温反应如果用干冰作为冷却剂,一般就会标注$-78\tc$,这在有机反应中很常见.}.
\paragraph{\ce{CO2}的制备与用途}
实验室制备\ce{CO2}可以用\ce{CaCO3}与稀\ce{HCl}反应得到:
\begin{center}
    \ce{CaCO3 + 2HCl -> CaCl2 + H2O + CO2}
\end{center}
工业上则是高温煅烧石灰石得到:
\begin{center}
    \ce{CaCO3 ->T[高温] CaO + CO2}
\end{center}
可以借由下面的两种方法实现\ce{CO2}的可逆回收与释放:
\begin{center}
    \ce{Na2CO3 + H2O + CO2 <=>T[冷却][加热] 2NaHCO3}\\
    \ce{2HO(CH2)2NH2 + H2O + CO2 <=>T[$25\sim65\tc$][$100\sim150\tc$] [HO(CH2)2NH3]2CO3}
\end{center}
有时,制备得到的\ce{CO2}中含有\ce{H2S}杂志,这可以通过近中性缓冲的\ce{KMnO4}的饱和溶液去除:
\begin{center}
    \ce{3H2S + 2KMnO4 + 2CO2 -> 3S + 2MnO2 + 2KHCO3 + 2H2O}
\end{center}

\indent \ce{CO2}最广泛的用途是作为冷冻剂.干冰主要用作冰淇淋,肉类和冷冻食品的致冷剂,并且用作一种方便的实验室冷冻剂和致冷剂.另一个主要用途是生产碳酸饮料\footnote{需要注意的是,碳酸饮料的酸性主要是其中添加的磷酸实现的.}.\ce{CO2}作为酸性气体可以中和碱性废水,也可以制造尿素:
\begin{center}
    \ce{CO2 + 2NH3 -> NH2COONH4}\\
    \ce{NH2COONH4 -> CO(NH2)2 + H2O}
\end{center}
\paragraph{\ce{CO2}的化学性质}
一般情况下,\ce{CO2}是含碳物质燃烧的最终产物.在较高的温度下,它会分解生成\ce{CO},因此高温反应一般不以\ce{CO2}作为产物写入方程式中.\\
\indent \ce{CO2}与格氏试剂\ce{RMgX}反应生成多一个碳的羧酸:
\begin{center}
    \ce{RMgX + CO2 -> RCOOMgX}
\end{center}

\indent 此外的最主要的性质就是\ce{CO2}的水溶液化学.我们单独将其写为一段.
\paragraph{\ce{CO2}的水溶液化学}
前面已经说过,\ce{CO2}可溶于水.溶于水的部分\ce{CO2}能与\ce{H2O}反应生成\ce{H2CO3}:
\begin{center}
    \ce{CO2(aq) + H2O(l) <=> H2CO3(aq)}\ \ \ $\dfrac{1}{K}=6\times10^2$
\end{center}
关于碳酸的性质,我们放在下一节描述.
\subsubsection{碳酸与碳酸盐}
\paragraph{\ce{H2CO3}的水溶液化学}
众所周知,碳酸\ce{H2CO3}是二元弱酸,两级电离平衡分别为
\begin{center}
    \ce{H2CO3(aq) + H2O(l) <=> H3O+(aq) + HCO3-(aq)}\\
    \ce{HCO3-(aq) + H2O(l) <=> H3O+(aq) + CO3^2-(aq)}
\end{center}
一般而言,碳酸的$K_{\text a1}$中的\ce{[H2CO3]}项是溶解的\ce{CO2}与\ce{H2CO3}的浓度之和.因此,碳酸的第一级表观电离常数和实际的电离常数分别为
\[K_{\text{a}1,\text{obs}}=\dfrac{\con{H3O+}\con{HCO3-}}{\con{CO2}+\con{H2CO3}}=4.45\times10^{-7}\]
\[K_{\text a1}=\dfrac{\con{H3O+}\con{HCO3-}}{\con{H2CO3}}=3.23\times10^{-4}\]
后面的数值就比较符合Pauling的预测了.第二级电离常数是可以测准的,为
\[K_{\text a2}=\dfrac{\con{H3O+}\con{CO3^2-}}{\con{HCO3-}}=4.84\times10^{-11}\]
\ce{H2CO3/HCO3-}缓冲对对于维持人体内环境的稳定有着举足轻重的作用,在自然界中也可以维持水体pH相对稳定.血浆的pH通常在$7.35\sim7.45$之间,因此人体的血液呈现弱碱性\footnote{一直以来有很多所谓人体是酸性还是碱性的说法,笔者认为有一定胡扯的成分.}.
\paragraph{高压碳酸盐化学}
在$20\text{ GPa}$的高压下加热\ce{SrO}与\ce{SrCO3}的混合物得到反钙钛矿型的\ce{Sr3CO5}:
\begin{center}
    \ce{2SrO + SrCO3 ->T[激光加热] Sr3CO5}
\end{center}
实际组成为\ce{[Sr^2+]3[CO4^4-][CO3^2-]},\ce{[CO4]^4-}占据(钙钛矿中,下同)\ce{Ca^2+}的位置,\ce{O^2-}占据\ce{Ti^4+}的位置,\ce{Sr^2+}占据\ce{O^2-}的位置.\\
\indent 在$82\sim138\text{ GPa}$的高压下,\ce{MgCO3}中的\ce{CO3^2-}发生三聚,形成六元环状的\ce{[C3O9]^6-}.\\
\indent 在$30\sim40\text{ GPa}$的高压下,用激光加热$\ce{MCO3}(\ce{M}=\ce{Ca},\ce{Sr})$和干冰的混合物可以得到\ce{MC2O5}.在\ce{CaC2O5}中存在与\ce{P4O10}结构相同的四聚离子\ce{[C4O10]^4-},而\ce{SrC2O5}中存在与\ce{N2O5}结构相同的\ce{[C2O5]^2-}.\\
\indent 以上提到的离子结构如下所示.
\begin{figure}[H]
    \centering
    \subfigure[\ce{[CO4]^4-}的结构]{
        \begin{minipage}[b]{.34\linewidth}
            \centering\includegraphics{picture/CO4^4-.eps}
        \end{minipage}
    }
    \subfigure[\ce{[C3O9]^6-}的结构]{
        \begin{minipage}[b]{.34\linewidth}
            \centering\includegraphics{picture/C3O96-.eps}
        \end{minipage}
    }
    \subfigure[\ce{[C4O10]^4-}的结构]{
        \begin{minipage}[b]{.34\linewidth}
            \centering\includegraphics{picture/C4O104-.eps}
        \end{minipage}
    }
    \subfigure[\ce{[C2O5]^2-}的结构]{
        \begin{minipage}[b]{.34\linewidth}
            \centering\includegraphics{picture/C2O52-.eps}
        \end{minipage}
    }\caption{高压下的碳酸根阴离子的结构}
\end{figure}
\subsubsection{碳的其它含氧酸及其盐}
\paragraph{草酸和草酸盐}
\paragraph{甲酸}
\subsection{碳的氢化物,卤化物和卤氧化物}
\subsection{碳的硫化物及其衍生物}
\subsubsection{碳的硫化物}
\paragraph{一硫化碳\ce{CS}}
与\ce{CO}不同,\ce{CS}是一种非常不稳定的物质,几乎只能作为反应中间体存在.它容易与其它VIA族元素和卤素反应生成\ce{CSSe},\ce{CSTe},\ce{CSX2}等物质.
\paragraph{}
\end{document}