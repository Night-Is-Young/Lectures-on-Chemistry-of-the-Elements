\documentclass{ctexart}
\usepackage{EC}
\begin{document}
\section{碳及其化合物}
\subsection{单质碳}
\subsubsection{石墨}
\paragraph{石墨的结构与性质}
众所周知,石墨具有六边形\ce{\{C6\}}环稠合的二维层状结构.
\chemfig{graphite-1}{1}{石墨的二维层状结构示意图}
根据层间重叠方式的不同,石墨有两种异构体,即六方$\alpha$-石墨和三方$\beta$-石墨,前者在自然界中占据主要.两者的晶体结构如下所示.
\bichemfig{alpha-graphite}{1}{六方$\alpha$-石墨的晶体结构示意图}{beta-graphite}{1}{三方$\beta$-石墨的晶体结构示意图}{石墨的两种异构体(黑点表示碳原子,虚线表示晶胞边界)}
从密堆积的角度描述这两种石墨的结构,六方$\alpha$-石墨的堆积层可以表示为$\cdots AB\overline{AB}AB\cdots$,而三方$\beta$-石墨的堆积层可以表示为$\cdots ABC\overline{ABC}ABC\cdots$.研磨$\alpha$-石墨可以使其转化为$\beta$-石墨,而加热$\beta$-石墨可以使其转化为$\alpha$-石墨.不完全的转化将得到湍层石墨,其中各层的堆积次序是随机的.\\
\indent 由于每一层均可以视作无限稠合的芳环,因而电子在层内是高度离域的,这使得石墨具有良好的导电性能,并且平行于层方向的电导率大约是垂直于层方向的$10^3$倍.\\
\indent 石墨的层间间距较大,只有不强的范德华力作用,使得其容易沿层平面滑动解理\footnote{\tbf{解理}(cleavage)又称\tbf{劈理},是矿物学和宝石学的常见术语,指的是矿物或宝石晶体在外力的作用下,沿一定的结晶学方向裂开成光滑平面的性质.这些光滑平面被称为\tbf{解理面}或\tbf{劈理面}.}.因此,石墨的硬度很低.
\paragraph{无定形碳与石墨的分布}
碳在史前就被认为是一种物质(炭,烟灰),而学界承认碳是一种元素却是18世纪几个实验的结论.大部分无定形碳都含有石墨的层状结构,然而晶体化程度很低,没有固定形状和周期性结构规律,同时经常掺杂很多其它元素.
\paragraph{石墨的分布,生产与用途}
晶状的石墨广泛分布于全世界,然而大多数以薄片的形式存在于硅酸盐矿石中,几乎没有经济价值.由于缺乏天然的来源,石墨在工业上主要通过焦炭与二氧化硅的反应制备:
\begin{center}
    \ce{SiO2 + 3C -> SiC}\\
    \ce{SiC(s) ->T[$2500\tc$] Si(g) + C(s,\text{石墨})}
\end{center}
石墨的主要用途如下:
\begin{enumerate}[label=\tbf{\arabic*.},topsep=0pt,parsep=0pt,itemsep=0pt,partopsep=0pt]
    \item \tbf{作为铅芯的主要成分制造铅笔}\\
        石墨的层状结构易剥落,可用于纸上书写.现代铅笔笔芯以石墨和黏土制造,石墨含量越高,铅芯越软,颜色越黑.\\
        铅笔之名源自其早期雏形为铅金属所制造,而后则又因欧洲中世纪时石墨被误以为是铅的一种,而有了\tbf{黑铅}这个名称,因此铅笔一词在各语言中流传使用而未修正.中国古代的铅笔事实上确为铅粉所制,并没有石墨.
    \item \tbf{惰性电极材料}\\
        石墨的优良的导电性和稳定性使得其可以作为电极材料.这一方案显然比铂电极更加便宜,并且不会受到卤素的侵蚀.因此,电解\ce{NaCl}水溶液制备\ce{Cl2}和\ce{NaOH}时通常采用石墨电极.
    \item \tbf{核反应堆的中子减速剂}
        石墨可以有效地减缓核反应堆里中子的速度,有效控制反应进行.石墨耐高温,纯度高,是迄今为止核反应堆中极为重要的原材料.
\end{enumerate}
\subsubsection{金刚石与蓝丝黛尔石}
\paragraph{金刚石的结构与性质}
在金刚石中,\ce{C}原子以$\text{sp}^3$杂化形式与相邻的\ce{C}原子成键,从而形成无限延伸的三维结构.金刚石也有两种异构体.大多数天然得到的或人工合成的金刚石均属立方晶系,因此一般而言我们所称的金刚石为立方金刚石.1967年在美国亚利桑那州的巴林杰陨石坑第一次鉴别出了六方金刚石,并以爱尔兰晶体学家D. K. Lonsdale\footnote{1929年, D. K. Lonsdale首次利用X射线衍射法证明了苯是平面的.1931年,她又首次利用Fourier变换光谱分析法解析了六氯苯的结构,为推动芳香性的研究做出了重大贡献.她是英国皇家化学会首次选入的两个女会士之一,伦敦学院大学的首位终身女教授,国际晶体学会第一位女主席,英国科学促迸学会首位女主席.}命名为蓝丝黛尔石.两者的结构如下所示.
\bichemfig{diamond-1}{0.1}{立方金刚石的晶体结构}{diamond-2}{0.1}{六方金刚石的晶体结构}{两种金刚石的晶体结构}
事实上,分别将闪锌矿和纤锌矿中的所有原子替换为碳原子即可分别得到立方金刚石和六方金刚石.\\
\indent 蓝丝黛尔石推测为流星上的石墨坠入地球时在高温高压下形成.它保留了石墨的平行六边形结构,但层间的碳原子成键.蓝丝黛尔石相较立方金刚石不稳定,这可能是因为其中存在船式六元环(即存在重叠式构象的碳原子)所致.有研究表明,蓝丝黛尔石具有比立方金刚石更高的硬度和更强的抗压能力.然而,天然存在的蓝丝黛尔石不纯,且并不完全为六方结构.\\
\indent 金刚石是已知硬度最高的天然物质.同时,它的化学性质也非常稳定,即使在纯\ce{O2}中也要加热到$720\tc$才能燃烧.
\paragraph{金刚石的合成与应用}
由于金刚石的硬度极高且导热性极高,因此常用于用于钻探,研磨工具之上,可以用来切削和刻划其他物质.自1955年通用电气发明通过高温高压处理石墨获得人造金刚石的技术后,人们已经可以利用气相沉积法等制成金刚石微粒,因此现在细小颗粒的人工合成钻石已经较同级天然钻石便宜.故此,天然钻石的工业价值已经完全消失,目前的主要用途已仅限于首饰与观赏.随着人工合成技术的成熟,合成钻石也进入了首饰市场,但总是遭到天然钻石公司的诋毁.
\end{document}