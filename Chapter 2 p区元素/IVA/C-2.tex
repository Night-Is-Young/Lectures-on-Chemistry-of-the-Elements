\documentclass{ctexart}
\usepackage{EC}
\begin{document}
\section{碳及其化合物}
\subsection{碳的氢化物,卤化物和卤氧化物}
\subsubsection{碳的氢化物}
关于各种烃类化合物的性质,在有机化学书中已经有详细的描述,这里就不再详细说明了.
\subsubsection{碳的卤化物}
我们在这里主要介绍四卤化碳的性质.\\
\indent 室温下\ce{CF4}是气体,\ce{CCl4}是液体,\ce{CBr4}和\ce{CI4}是固体.\\
\indent \ce{CF4}异常稳定,是含碳化合物氟化的最终产物.它可以由下面的反应得到:
\begin{center}
    \ce{C + 2F2 -> CF4}\\
    \ce{4AgF + CCl4 ->T[$573\K$] 4AgCl + CF4}
\end{center}
尽管\ce{CF4}的水解作用是放热的:
\begin{center}
    \ce{CF4(g) + 2H2O(g) -> CO2(g) + 4HF(g)}\ \ \ $\Delta_\text rG_\text m^\ominus=-151\kJm$
\end{center}
但这一水解反应在动力学上是极为缓慢的.\ce{CF4}在动力学上非常稳定,这也意味着聚四氟乙烯的稳定性主要是动力学上的稳定性.\\
\indent 碳单质和\ce{Cl2}不能直接化合,因此制备\ce{CCl4}通常需要\ce{CS2}与\ce{Cl2}反应得到,这在介绍\ce{CS2}的性质时已经提到过:
\begin{center}
    \ce{CS2 + 3Cl2 ->T[cat.] CCl4 + S2Cl2}
\end{center}
此外,用甲烷的氯化也可以得到\ce{CCl4}:
\begin{center}
    \ce{CH4 + 4Cl2 -> CCl4 + 4HCl}
\end{center}
\ce{CCl4}是实验室常用的溶剂,它可以与乙醇等大部分有机溶剂混溶.\ce{CCl4}在常温下不活泼,不与酸或碱发生反应,在水中也难以溶解,但对一些金属单质(例如铝,铁)有明显的腐蚀作用.\ce{CCl4}有麻醉的作用;\ce{CCl4}也可以用作灭火剂.\\
\indent \ce{CBr4}可以通过甲烷的溴化得到,\ce{CI4}可以通过下面的反应得到:
\begin{center}
    \ce{CCl4 + 4C2H5I -> CI4 + 4C2H5Cl}
\end{center}

\subsection{氰及其衍生物}
\subsubsection{关于氰的基本介绍}
所谓氰,除去指氰气外,通常而言指\ce{-CN}基团或\ce{CN-}离子.它得名于形成深蓝色色素的性质,例如与铁盐形成普鲁士蓝,其希腊语为$\kappa\acute{\nu}\alpha\eta o\varsigma$,对应的英文为$cyanos$,意为“深蓝”.\\
\indent 与氰相关的另一个重要的概念是\tbf{拟卤素}.通常,类似\ce{CN,OCN,SCN}等含有氰基的基团在一般的反应中不会发生改变,并且和卤素一样是一价的.这些基团能形成阴离子\ce{X-},氢化物(通常是酸)\ce{HX},有时形成中性分子\ce{X2},\ce{XY}等等.种种性质表明它们与卤素有一定的类似,因此称它们为拟卤素.\\
\indent 此外,可以发现\ce{CN-}和\ce{CO},\ce{N2}和\ce{NO+}是等电子体.与此类似的,\ce{OCN-}与\ce{CO2},\ce{N3^-}是等电子体.\\
\indent 现在,我们介绍氰及其衍生物中主要的几种.
\subsubsection{氰\ce{(CN)2}}
氰是碳的最简单的氮化物,由两个\ce{CN}基团连接而成.
\paragraph{\ce{(CN)2}的物理性质}
\ce{(CN)2}是无色的苦杏仁味气体,有剧毒.\ce{(CN)2}可溶于水,乙醇,乙醚.\\
\indent \ce{(CN)2}可燃,燃烧时呈桃红色火焰,边缘侧带蓝色.\ce{(CN)2}在纯氧中的燃烧可以达到$4525\tc$以上的高温,仅次于二氰乙炔\ce{C2(CN)2}.
\paragraph{\ce{(CN)2}的制备}
实验室中通常可以采取\ce{KCN}与\ce{CuSO4}的反应制取\ce{(CN)2}.这可以类比\ce{Cu^2+}与\ce{I-}的反应,它们都生成了卤素/拟卤素“单质”和\ce{Cu^I}的难溶盐:
\begin{center}
    \ce{2CuSO4 + 4KCN -> (CN)2 + 2CuCN + 2K2SO4}
\end{center}
该反应的产率在$80\%$左右,主要的副产物为\ce{CO2}.副产物\ce{CuCN}可以进一步用\ce{FeCl3}的热水溶液氧化以释放其中的\ce{CN-}:
\begin{center}
    \ce{2CuCN + 2FeCl3 ->T[$\Delta$] (CN)2 + 2CuCl + 2FeCl2}
\end{center}

\indent 工业上主要采取氰化氢的催化氧化制得:
\begin{center}
    \ce{4HCN + O2 ->T[\ce{Ag}] 2H2O + 2(CN)2}\\
    \ce{2HCN + Cl2 ->T[\ce{CaO/SiO2}] 2HCl + (CN)2}
\end{center}

\indent 此外,对草酰胺脱水也得到\ce{(CN)2}.不过这不是主要的制备方法,但可以据此而认为氰是草酸衍生的腈.
\begin{center}
    \ce{H2NCOCONH2 ->T[脱水] (CN)2 + 2H2O}
\end{center}
\paragraph{\ce{(CN)2}的性质与反应}
纯的\ce{(CN)2}具有相当高的热稳定性,但痕量杂质的存在将使得其在$300\sim500\tc$时发生聚合而生成顺氰,其中具有吡嗪环并接的长链结构:
\chemfig{(CN)n}{1}{顺氰的结构}
\ce{(CN)2}在一般的溶剂中比较稳定,但在碱性水溶液中发生类似卤素单质的歧化,生成氰根离子\ce{CN-}和氧氰阴离子\ce{OCN-}:
\begin{center}
    \ce{(CN)2 + 2OH- -> CN- + OCN- + H2O}
\end{center}
\subsubsection{氰化氢\ce{HCN}}
\paragraph{\ce{HCN}的物理性质}
\ce{HCN}是无色的杏仁味气体\footnote{这种气味主要是因为杏仁核中的苦杏仁苷水解所产生的\ce{HCN}.此事在Clayden等所著的有机化学书中有详细的记载.}\footnote{据说能否明显地闻到此气味却决于个人的基因.},剧毒且致命.\ce{HCN}的熔点为$-13.4\tc$,沸点为$25.6\tc$,因而在标准状态下是液体.
\paragraph{\ce{HCN}的制备}
以前,制备\ce{HCN}主要靠\ce{NaCN}和\ce{Ca(CN)2}等氰化物盐的酸化.甲酸铵的脱水也可以得到\ce{HCN}:
\begin{center}
    \ce{HCOONH4 -> HCN + 2H2O}
\end{center}
现在通常通过催化反应合成\ce{HCN}:
\begin{center}
    \tbf{Andrussow过程}:\ \ce{2CH4 + 2NH3 + 3O2 ->T[\ce{Pt/Rh/Ir}][$\Delta$] 2HCN + 6H2O}\\
    \tbf{Degussa过程}:\ \ce{CH4 + NH3 ->T[\ce{Pt}][$\Delta$] HCN + 3H2}
\end{center}
\paragraph{\ce{HCN}的性质,反应与应用}
\ce{HCN}在水中是弱酸:
\begin{center}
    \ce{HCN(aq) + H2O(l) <=> H3O+(aq) + CN-(aq)}\ \ \ $K_{\text a}=2.1\times10^{-9}$
\end{center}
它也是中等的还原剂.
\begin{center}
    \ce{(CN)2 + 2H+ + 2e- -> 2HCN}\ \ \ $E^\ominus=+0.37\text{ V}$
\end{center}
用较强的还原剂,例如\ce{Pd/H2}即可将\ce{HCN}还原为\ce{MeNH2}.\\
\indent 液态\ce{HCN}容易发生聚合.三聚体和四聚体的结构如下所示.
\bichemfig{(HCN)3}{1}{\ce{HCN}的三聚体结构}{(HCN)4}{1}{\ce{HCN}的四聚体的结构}{\ce{HCN}的多聚体结构}
\indent 其它的性质主要是\ce{CN-}的性质.因此,我们放到下一小节介绍.
\subsubsection{氰根阴离子\ce{CN-}与氰化物}
\paragraph{\ce{CN-}的毒性}
应当说,绝大部分氰类化合物的毒性都来源于\ce{CN-}.它与细胞色素c氧化酶中的\ce{Fe}原子结合,抑制其活性,从而干扰正常有氧呼吸的过程.高度依赖有氧呼吸的组织,例如心脏和中枢神经系统,受此影响最大.\\
\indent 氰化物中毒的临床症状包括:中毒者血液的pH值在中毒后两至三分钟内急剧下降\footnote{这应当是因为细胞只能采取无氧呼吸,产生大量丙酮酸的缘故.};缺氧窒息;身体散发大量类似苦杏仁味的气味;严重昏迷及面部发紫;即使痊愈后,大部分中毒者的脑部和心脏一般都已受永久性伤害.\\
\indent 在众多侦探和悬疑小说中,通常用于毒杀他人的物质就是氢氰酸.综合而言,它容易得到,并且毒性足够强烈.现在,氰化物都列入严格管制,以免不法分子用它危害他人生命.
\paragraph{氰化物的制备}
主要需要制备的简单氰化物通常是碱金属和碱土金属的氰化物.\ce{NaCN}的早期的工业制备方法是将\ce{NaNH2}与碳反应得到:
\begin{center}
    \ce{NaNH2 + C ->T[$\Delta$] NaCN + H2}
\end{center}
现在则由氰氨化钙,碳和碳酸钠反应得到:
\begin{center}
    \ce{CaCN2 + Na2CO3 + C -> 2NaCN + CaCO3}
\end{center}
\ce{NaCN}大量地用于配制金属电镀液.
\paragraph{\ce{CN-}的配位化学与反应}
和它的等电子体\ce{CO}一样,\ce{CN-}也有着十分强的配位能力,其配合物也数目众多,并且一般而言用\ce{C}作为配位原子.我们在这里简单地对\ce{CN-}的配合物进行介绍,剩下的内容放到各自对应的中心元素中.\\
\indent 氰化物曾经被大量用于金和银的开采中,它有助于溶解这些金属,从而与其他固体分离.反应的方程式如下:
\begin{center}
    \ce{4Au + 8NaCN + O2 + 2H2O -> 4Na[Au(CN)2] + 4NaOH}\\
    \ce{Ag2S + 4NaCN + H2O -> 2Na[Ag(CN)2] + NaSH + NaOH}
\end{center}
其中\ce{[Au(CN)2]-}和\ce{[Ag(CN)2]-}都是直线形的阴离子.由于氰化物的剧毒性,这种方法对环境的破坏是相当大的,因此已经被禁止使用.然而在某些手工提取电子产品中的金的视频中,似乎仍然有人采用这种不规范的方法.\\
\indent 除此之外,\ce{CN-}也能作为桥连配体.在\ce{AgCN}和\ce{AuCN}中,每个\ce{CN-}分别用\ce{C}和\ce{N}对两个\ce{M+}配位,形成无限延伸的一维长链.
\bichemfig{M(CN)2-}{1}{\ce{[M(CN)2]-}的结构}{MCN}{1}{\ce{MCN}中的一维长链}{$\ce{M}(\ce{M}=\ce{Ag},\ce{Au})$的\ce{CN-}配合物}
这种桥连的形式还存在于很多物质中,例如\ce{[Co2(NH3)10(CN)]^5+},\ce{[Au(C3H7)2(CN)]4}和层型结构的\ce{Pd(CN)2}等.这些物质的结构如下所示.
\chemfig{Co2(NH3)10(CN)5+}{1}{\ce{[Co2(NH3)10(CN)]^5+}的结构}
\bichemfig{(AuR2CN)4}{1}{\ce{[Au(C3H7)2(CN)]4}}{Pd(CN)2}{1}{\ce{Pd(CN)2}的结构}{\ce{CN-}桥连的平面四方配合物}
在\ce{CuCN.NH3}中有\ce{CN-}桥连的二维结构(除去对\ce{Cu}配位的\ce{NH3}),但与上面不同的是\ce{C}同时对两个\ce{Cu^I}配位,同时这两个\ce{Cu^I}之间还有类似亲金作用的亲铜作用.这一层状结构如下所示.
\chemfig{CuCN}{1}{\ce{CuCN.NH3}的二维层}
\subsubsection{氰的卤化物\ce{XCN}}
氰与四种卤素的化合物都已经制得.\ce{ClCN}和\ce{BrCN}可以由对应的卤素与碱金属氰化物反应得到:
\begin{center}
    \ce{X2 + MCN -> MCl + XCN}
\end{center}
将干燥的\ce{Hg(CN)2}与\ce{I2}反应即可得到\ce{ICN}:
\begin{center}
    \ce{Hg(CN)2 + 2I2 -> 2ICN + HgI2}
\end{center}
而\ce{FCN}则由\ce{(FCN)3}的分解得到.后者可以由\ce{NaF}与\ce{(ClCN)3}反应得到.事实上,所有四种\ce{XCN}都可以发生三聚形成含有六元环的\ce{(XCN)3},其结构如下:
\chemfig{(XCN)3}{1}{\ce{(XCN)3}的结构}
\ce{XCN}可以发生取代反应,得到各种含有氰基的化合物.以下是一些例子:
\begin{center}
    \ce{XCN + NaN3 -> N3CN + NaX}\\
    \ce{XCN + 2NaOH + NaX + NaOCN + H2O}\\
    \ce{ClCN + AgOCN -> NCOCN + AgCl}
\end{center}
\subsubsection{氨基氰\ce{H2NCN}及其衍生物}
\paragraph{氨基氰的制备与结构}
氨基氰(即氰胺)可以由氰氨化钙按如下方式制备得到:
\begin{center}
    \ce{2CaCN2 + 2H2O -> Ca(HNCN)2 + Ca(OH)2}\\
    \ce{Ca(HNCN)2 + CO2 + H2O -> CaCO3 + 2H2NCN}
\end{center}
氰胺是平面型分子,其结构如下:
\chemfig{H2NCN}{1}{氰胺分子的结构}
氰胺也有一个不稳定的异构体,即二亚胺\ce{HN=C=NH}.有机化学中常用的脱水剂DCC(即二环己基二亚胺)就是二亚胺的衍生物.
\paragraph{氨基氰的反应}
\indent 常用的氰胺是以二聚体的形式存在的.这二聚体也有一个异构体,如下所示.
\bichemfig{(H2NCN)2-1}{1}{\ce{H2NCN}的二聚体I}{(H2NCN)2-2}{1}{\ce{H2NCN}的二聚体II}{\ce{H2NCN}的二聚体}
氰胺可以发生三聚.三聚氰胺常用作制备树脂,用作阻燃剂,减水剂和甲醛清洁剂等等.不法商家曾经将三聚氰胺加入奶粉中以提高氮含量,进而让检测出的蛋白质含量更高\footnote{这就是曾经轰动全国的著名的三鹿奶粉事件.}.由于三聚氰胺对神经系统和泌尿系统的毒害作用,这对婴幼儿有极其恶劣的健康影响.
\chemfig{(H2NCN)3}{1}{三聚氰胺的结构}
氰胺在酸性溶液中可以和\ce{HNO2}反应,产生异氰酸:
\begin{center}
    \ce{H2NCN + HNO2 -> HNCO + N2 + H2O}
\end{center}
氰胺也可以和尿素发生加和反应:
\begin{center}
    \ce{H2NCN + (NH)2CO -> H2N-C(NH)-NH-CO-NH2}
\end{center}
\subsubsection{氰酸HOCN,异氰酸HNCO和雷酸HONC}
\paragraph{\ce{HOCN},\ce{HNCO}和\ce{HONC}的结构}
氰酸和异氰酸互为互变异构体.除此之外,它们还有一种键合异构体,即雷酸.它们的结构如下所示:
\begin{figure}[H]
    \centering
    \subfigure[氰酸\ce{HOCN}的结构]{
        \begin{minipage}[b]{.3\linewidth}
            \centering\includegraphics{picture/HOCN.eps}
        \end{minipage}
    }
    \subfigure[异氰酸\ce{HNCO}的结构]{
        \begin{minipage}[b]{.3\linewidth}
            \centering\includegraphics{picture/HNCO.eps}
        \end{minipage}
    }
    \subfigure[雷酸\ce{HCNO}的结构]{
        \begin{minipage}[b]{.3\linewidth}
            \centering\includegraphics{picture/HCNO.eps}
        \end{minipage}
    }\caption{氰酸,异氰酸和雷酸的结构}
\end{figure}
和氰的卤化物一样,氰酸和异氰酸也有其三聚体.大多数地方都将其画成与\ce{(XCN)3}一样的具有三嗪结构的环,但光谱和晶体衍射数据都表明三聚体事实上绝大部分以三酮的形式存在.在溶液中,也少量地存在单羟基的形式.这可能是酰胺结构的强稳定性导致的.
\bichemfig{(HOCN)3}{1}{三聚氰酸可能存在的形式}{(HNCO)3}{1}{三聚氰酸实际上的存在形式}{三聚氰酸的结构}
\paragraph{\ce{HNCO},\ce{HONC}及其三聚体的性质}
氰酸和异氰酸都是有毒的.\\
\indent 一个容易引起混淆的问题是氰酸和异氰酸的稳定性.尽管从命名上说,带异字的异构体通常是不稳定的,但事实上异氰酸才是更稳定的异构体.两者之间具有明显的能量差距\footnote{\textit{J. Chem. Phys.}  \textit{120}, 11586(\tbf{2004}); doi: 10.1063/1.1707013.}:
\begin{center}
    \ce{HOCN <=> HNCO}\ \ \ $\Delta H^{\ominus}(298\K)=-103.4\kJm$
\end{center}

\indent 加热分解三聚氰酸可以得到这两种异构体,但绝大部分为\ce{HNCO}.酸化\ce{OCN-}的盐也将得到异氰酸.将尿素加热可以按照以下方法制得异氰酸:
\begin{center}
    \ce{CO(NH2)2 ->T[$\Delta$] HNCO + NH3}
\end{center}

\indent 1828年,F. Wöhler报道了首例从无机物合成有机物尿素的办法,即异氰酸铵的热异构化:
\begin{center}
    \ce{NH4Cl + AgNCO -> NH4NCO + AgCl}\\
    \ce{NH4NCO ->T[$\Delta$] CO(NH2)2}
\end{center}
两者在溶液中处于平衡,大约有$5\%$以\ce{NH4NCO}的形式存在.这是首例以无机物合成有机物的反应,它打破了当时盛行的生命力论,证明了有机化合物并不只能由生命过程得到,有机化学从此开始蓬勃发展.\\
\indent 异氰酸在水中发生水解,其中可能经过了\ce{H2NCOOH}这一中间体:
\begin{center}
    \ce{HNCO + H2O -> [NH2COOH] -> NH3 + CO2}
\end{center}
\indent 三聚氰酸和三聚氰胺一样,本身的毒性是较低的.然而,混合两者可以形成不溶的,具有二维层状结构的氰尿酸三聚氰胺,其结构如下:
\chemfig{(HNCO)3(H2NCN)3}{1}{氰尿酸三聚氰胺的层状结构}
这是超分子组装的一个经典例子.单独摄入三聚氰酸对人体没有明显的毒害作用,但单独摄入三聚氰胺或两者一起摄入时(三聚氰胺在人体中也可以水解为三聚氰酸),由于人体无法利用这两种物质,因此它们将被运往肾脏并准备排出体外.由于肾脏的浓缩作用,这两种物质在血液中的含量增加,超过临界浓度,在肾小管中形成大量黄色粒状的氰尿酸三聚氰胺固体,致使肾小管的物理阻塞,尿液无法排出,从而最终导致肾脏衰竭.
\paragraph{\ce{HCNO}及其盐的性质}
E. C. Howard在1800年就发现了雷酸银\ce{AgCNO}的存在.1966年,由稀的雷酸钠溶液酸化后小心地蒸出了雷酸.雷酸也是有毒的.\\
\indent 对于雷酸的研究很少,它和它的盐(主要是\ce{AgCNO}和\ce{Hg(CNO)2})都是不稳定而易爆的.这两种盐以前经常用作雷管的起爆药,按如下的方式爆炸:
\begin{center}
    \ce{2AgCNO -> 2Ag + 2CO + N2}\\
    \ce{Hg(CNO)2 -> Hg + 2CO + N2}
\end{center}
雷酸银\ce{AgCNO}是一种起爆药,由于它对热,压力和电极度敏感,因此它几乎没有实用价值,以前用的以雷酸汞为主.随着\ce{AgCNO}的聚集,它变得越来越敏感,即使量很少.一堆水下的雷酸银,下落羽毛的轻触,一滴水的撞击,或一次小的静电放电都能够将其引爆.不可能聚集起较大量雷酸银,因为其自身重量将会使其爆炸.\\
\indent 在19世纪20年代,J. F. von Liebig和F. Wöhler先后合成了雷酸银\ce{AgCNO}和异氰酸银\ce{AgNCO},然后引起了它们是否为同一物质的争论.最后,J. J. Berzelius引出了同分异构体的概念以解释这两种物质截然不同的性质.时至今日,同分异构体已经成为化学中的重要概念.\\
\indent \ce{AgNCO}的晶体结构比较特殊.这种六方晶系的晶体中有两种环:\ce{[Ag(CNO)]2}和\ce{[Ag(CNO)6]},其晶胞如下所示.
\bichemfig{AgCNO-1}{0.1}{\ce{AgCNO}的晶胞示意图}{AgCNO-2}{0.1}{\ce{AgCNO}的晶胞沿$c$轴的投影图}{\ce{AgCNO}的晶体结构}
这两种环的示意图如下.
\bichemfig{AgCNO-3}{1}{\ce{[Ag(CNO)6]}的晶胞示意图}{AgCNO-4}{1}{\ce{[Ag(CNO)]2}的示意图}{\ce{AgCNO}中的两种环}
\subsubsection{硫氰及其衍生物}
\paragraph{硫氰及其衍生物的制备}
将含\ce{CN-}的盐与硫单质共熔或共热即可得到相应的硫氰酸盐:
\begin{center}
    \ce{8KCN + S8 -> 8KSCN}
\end{center}
然后用\ce{KHSO4}与其反应即可得到\ce{HSCN}:
\begin{center}
    \ce{KSCN + KHSO4 -> K2SO4 + HSCN}
\end{center}
将\ce{AgSCN}悬浮于乙醚中并用卤素单质氧化,即可得到\ce{(SCN)2}:
\begin{center}
    \ce{2AgSCN + Br2 -> 2AgBr + (SCN)2}
\end{center}
\paragraph{硫氰及其衍生物的结构,性质与反应}
硫氰\ce{(SCN)2}是折线形的分子,其中含有一根\ce{S-S}键:
\chemfig{(SCN)2}{1}{\ce{(SCN)2}的结构}
\ce{(SCN)2}为浅黄色固体,熔点为$270\K$.在高于熔点的温度下,它会不可逆地发生聚合反应生成砖红色的\ce{(SCN)_n}.这一聚合物有多种可能的结构,如下所示.
\begin{figure}[H]
    \centering
    \subfigure[\ce{(SCN)_n}的可能结构I]{
        \begin{minipage}[b]{.35\linewidth}
            \centering\includegraphics{picture/(SCN)n-1.eps}
        \end{minipage}
    }
    \subfigure[\ce{(SCN)_n}的可能结构II]{
        \begin{minipage}[b]{.35\linewidth}
            \centering\includegraphics{picture/(SCN)n-2.eps}
        \end{minipage}
    }
    \subfigure[\ce{(SCN)_n}的可能结构III]{
        \begin{minipage}[b]{.8\linewidth}
            \centering\includegraphics{picture/(SCN)n-3.eps}
        \end{minipage}
    }\caption{\ce{(SCN)_n}的可能结构}
\end{figure}
\ce{(SCN)2}易溶于水,但在水中很快地发生歧化反应:
\begin{center}
    \ce{(SCN)2 + H2O -> HOSCN + HSCN}
\end{center}
加热后,\ce{HOSCN}也能进一步发生分解:
\begin{center}
    \ce{3HOSCN + H2O -> 2HSCN + HCN + H2SO4}
\end{center}
硫氰具有一定的氧化性,在水溶液中可以发生与\ce{I2}类似的反应:
\begin{center}
    \ce{(SCN)2 + 2S2O3^2- -> S4O6^2- + 2SCN-}\\
    \ce{(SCN)2 + H2S -> 2HSCN + S}
\end{center}

\indent 硫氰酸是无色的易挥发的液体,在常温下迅速分解.硫氰酸在水溶液中是中强酸:
\begin{center}
    \ce{HSCN(aq) + H2O(l) <=> H3O+(aq) + SCN-(aq)}\ \ \ $\text pK_{\text a}=1.1$
\end{center}
不难想到,事实上\ce{HSCN}的异构体异硫氰酸\ce{HNCS}更加稳定.但习惯上可以写成\ce{HSCN}.
\paragraph{\ce{SCN-}的配位形式}
我们熟知\ce{SCN-}具有下面两种共振式:
\chemfig{SCN-}{1}{\ce{SCN-}的共振式}
计算结果表明\ce{S},\ce{C}和\ce{N}上带的电荷分别为$-0.463,+0.055$和$-0.592$.这表明负电中心在\ce{N}原子上;而HOMO系数最大处不难知道在\ce{S}原子上.\ce{SCN-}作为配体的配位原子选择和这一事实密切相关.\\
\indent 定性而言,\ce{SCN-}的软的一端为\ce{S},硬的一端为\ce{N},可以用软硬酸碱理论解释具体的配位形式.对于电荷高,半径小的离子,例如\ce{Fe^3+},倾向于形成\ce{N}端配位的配合物;对于高周期低价态的金属离子,例如\ce{Ag+}等等,倾向于形成\ce{S}端配位的配合物.\\
\indent \tbf{一般而言},用\ce{N}端配位时为直线型,用\ce{S}端配位时为折线形.但有时受到其它配体位阻或电子效应的影响,也会出现不寻常的结构.\\
\indent 有时,配体的性质也会影响配位原子和配位形式.例如,\ce{Co^{III}}的配合物\ce{[Co(NH3)5(NCS)]^2+}以\ce{N}端配位,构型为直线型;而与其相似的\ce{[Co(CN)5(SCN)]^3-}以\ce{S}端配位,构型为折线形.
\bichemfig{[Co(NH3)5(NCS)]2+}{1}{ce{[Co(NH3)5(NCS)]2+}的结构}{[Co(CN)5(SCN)]3-}{1}{\ce{[Co(CN)5(SCN)]^3-}的结构}{两种\ce{Co^{III}-SCN}配合物的结构}
\subsubsection{硒氰及其衍生物}
与硫相似,硒可以溶于氰化物的溶液中,形成硒氰酸盐:
\begin{center}
    \ce{NaCN + Se -> NaSeCN}
\end{center}
同样地,用\ce{I2}氧化悬浮在有机溶剂(如乙醚,\ce{CCl4})中的\ce{AgSeCN}可以得到\ce{(SeCN)2}:
\begin{center}
    \ce{2AgSeCN + I2 -> (SeCN)2 + 2AgI}
\end{center}
\ce{(SeCN)2}也容易发生聚合.在乙醇溶剂中,它形成以下的结构:
\chemfig{(SeCN)n}{1}{\ce{(SeCN)_n}的结构}
\ce{(SeCN)2}容易水解生成亚硒酸:
\begin{center}
    \ce{2(SeCN)2 + 3H2O -> H2SeO3 + HCN + 2HSeCN}
\end{center}
在\ce{CS2}中加热\ce{(SeCN)2}可使其发生分解:
\begin{center}
    \ce{2(SeCN)2 ->T[$\Delta$] Se(CN)2 + Se(SeCN)2}
\end{center}
二氰化硒可以按如下方式水解:
\begin{center}
    \ce{4Se(CN)2 + 3H2O -> H2SeO3 + 6HCN + Se(SeCN)2}
\end{center}
\subsection{碳的其它氮化物和磷化物}
\subsubsection{石墨氮化碳}
将石墨层的结构稍加修改,就能得到化学式为\ce{C3N4}的石墨氮化碳.和石墨一样,\ce{C3N4}也具有无限延伸的二维层状结构,并且有以下两种异构体:
\bichemfig{C3N4-1}{0.9}{\ce{C3N4}的层状结构I}{C3N4-2}{0.9}{\ce{C3N4}的层状结构II}{\ce{C3N4}的层状结构}
如果你仔细观察,就可以发现这些层状结构是由$\dfrac{n(n+1)}{2}$个均三嗪并接成的边长为$n$的正三角形单元和连接这些单元的\ce{N}原子组成的.然而,当$n>2$时,由于成键的限制将不能画出类似的结构.因此\ce{C3N4}似乎仅有上面两种比较合理的异构体.\\
\indent 石墨氮化碳作为一种新型二维材料,在催化和能源等领域有着重要用途.
\subsubsection{尿素及其衍生物}
这一节内容事实上和各种氰类化合物有一定的关系,但毕竟尿素中没有氰基,所以单独列在此处讲.
\paragraph{尿素的结构}
\subparagraph{尿素的结构}
尿素的化学式为\ce{CO(NH2)2},看起来是一个具有两个酰胺基团的平面型分子.事实上,在尿素晶体中,每个\ce{CO(NH2)2}分子确实是平面型的,并且存在无限延伸的氢键网络.
\bichemfig{CO(NH2)2-1}{1}{\ce{CO(NH2)2}晶体中的一维长链}{CO(NH2)2-2}{1}{\ce{CO(NH2)2}晶体中长链的连接形式}{\ce{CO(NH2)2}的氢键网络示意图}
\bichemfig{CO(NH2)2-3}{0.1}{\ce{CO(NH2)2}的晶胞示意图}{CO(NH2)2-4}{0.1}{\ce{CO(NH2)2}的晶胞沿$c$轴的投影图}{\ce{CO(NH2)2}的晶体结构示意图}
然而,尿素分子在气相中是非平面型的\footnote{\textit{J. Chem. Phys.}, \tbf{1989}, \textit{89}, 4560-4565; doi: 10.1021/j100267a029.}.笔者在这里想到了一个相关的问题.
\subparagraph{再议肽键的结构与成键方式}
在初学有机化学时,我们会学到大部分酰胺基团都是平面型的,其中的\ce{N}原子的杂化形式为$\text{sp}^2$.对于肽键(即羧酸与伯胺形成的酰胺键),它存在如下几个重要的结构特征:
\begin{enumerate}[label=\tbf{\arabic*.},topsep=0pt,parsep=0pt,itemsep=0pt,partopsep=0pt]
    \item 酰胺基团是平面型的,具有两种构型,构型的旋转能垒约为$15\sim20\text{ kcal/mol}$.
    \item 对于肽键,它明显地偏好Z构型,即\ce{C=O}键与\ce{N-H}键处于反式的构型.
    \item 肽键通常有强烈的形成氢键的倾向.这也是蛋白质的复杂构象得以稳定的基础.
\end{enumerate}
过去,我们通常会使用如下的共振式\tbf{B}解释这一结构,其中\ce{C}与\ce{N}成双键.
\chemfig{RCONHR-1}{1}{共振论预测的\ce{RCONHR}的主要共振式}
然而,计算化学的结果\footnote{\textit{Acc. Chem. Res.}, \tbf{1999}, \textit{32}, 922-929; doi: 10.1021/ar990008h.}部分否定了共振的解释.研究人员分别计算了酰胺采取平面型和交叉型(即将\ce{C-N}键旋转$90^\circ$的结构)的结构数据,发现旋转至不共面后,尽管\ce{C-N}键确实变长了,\ce{C-O}键的长度几乎没有任何变化.如果假定的共轭结构是正确的话,那么旋转\ce{C-N}键将阻止共振式\tbf{B}的形成,使得\ce{C-O}键变短,而结果却与其矛盾.另外,对酰胺基团电荷分布的计算结果表明,其中的\ce{N}带的正电荷并不比一般的氨基上多,\ce{O}带的负电荷也并不比一般的酮羰基上多.这意味着共振式\tbf{B}的电荷分布也是有一定问题的.\\
\indent 于是,他们提出了一个更合理的结构上的解释:酰胺的平面构型是由偶极相互作用控制的.
\chemfig{RCONHR-3}{1}{\ce{RCONHR}中的偶极相互作用}
由于\ce{C-O}键和\ce{N-H}键都具有明显的偶极矩,而偶极相互作用促使它们采取同样的方向排列.这也解释了肽键对Z构型的强烈倾向的现象.这一结论对应的主要共振式如下所示.
\chemfig{RCONHR-2}{1}{结合计算化学结果预测的\ce{RCONHR}的主要共振式}
尽管这一理论似乎对解释肽键之外的酰胺的结构稍有不足,但毕竟是对简单分子的结构的更符合事实的描述方式.笔者建议你在解释酰胺的结构时仍然主要采用共振论,但也许心里要明白偶极相互作用才是主要的(至少对于肽键而言).
\paragraph{尿素的制备,性质与反应}
尿素的首次合成由加热\ce{NH4NCO}而得到,这在氰酸及其盐一节已经介绍过.工业上制备尿素通过以下的办法:
\begin{center}
    \ce{2NH3(l) + CO2(s) -> H2NCOONH4}\\
    \ce{H2NCOONH4 ->T[$\Delta$] CO(NH2)2}
\end{center}
尿素作为一种高效的氮肥广泛地用于农业生产中.\\
\indent 加热尿素即可缩合得到双缩脲\ce{H2NCONHCONH2}:
\begin{center}
    \ce{2CO(NH2)2 ->T[$180\tc$] H2NCONHCONH2 + NH3}
\end{center}
双缩脲最重要的性质是在碱性溶液中与\ce{Cu^{II}}形成紫红色的配合物\ce{[Cu(C2N3O2H3)2]^2-},这一反应非常灵敏.
\bichemfig{C2N3O2H5}{1}{双缩脲\ce{H2NCONHCONH2}的结构}{[Cu(C2N3O2H3)2]2-}{1}{\ce{[Cu(C2N3O2H3)2]^2-}的结构}{双缩脲及其配合物的结构}
由于蛋白质中有类似的结构,因此\ce{Cu^{II}}也能与蛋白质发生明显的显色反应.
\chemfig{CuL2}{1}{蛋白质与\ce{Cu^{II}}的配合物}
由此,人们发明了双缩脲试剂.它是一个碱性的含铜试液,呈蓝色,由$1\%$强碱(\ce{NaOH}或\ce{KOH}等),几滴$1\%$\ce{CuSO4}和酒石酸钾钠配制.当底物中含有肽键(多肽)时,试液中的铜与多肽配位,配合物呈紫色.使用双缩脲试剂时,先将双缩脲试剂A(即\ce{NaOH}等碱的溶液)加入待测试液,然后加入双缩脲试剂B(即\ce{CuSO4}和酒石酸钠钾的混合溶液).如果试剂显紫红色,说明待测液中有多肽结构.唯一需要注意的是,二肽只有一个酰胺基,不与双缩脲试剂反应.
\paragraph{硫脲的结构,性质与反应}
\subsubsection{碳的磷化物及其衍生物}
1894年,Shober和Spanutius报道利用\ce{NaPH2}和\ce{CO}的反应制得\ce{NaCP},然而缺乏有力证据\footnote{在报告中,作者作了以下总结:我们不得不暂时停止这项研究,但在将来我们会再次讨论这个问题.目前尚不清楚究竟是什么阻碍了两人继续他们的研究(Grügtzmacher在综述中写道:我们怀疑,在19世纪后期,由好奇心驱动的科学研究可能和今天一样难以获得资金),}.直到Grügtzmacher等人在2011年重复该实验,才证明了这一反应得到的并非\ce{NaCP}而是\ce{NaOCP}.这一离子的制备和反应也是磷碳化学中一个有趣而重要的部分\footnote{\textit{Angew. Chem. Int. Ed.}, \tbf{2018}, \textit{57}, 16968-16994; https://doi.org/10.1002/anie.201803888.}.
\paragraph{\ce{PCO-}的制备}
1992年,Becker等人将\ce{LHDMS}的类似物\ce{LiP(TMS)2}与碳酸二甲酯反应,首次得到了\ce{PCO-}的盐:
\begin{center}
    \ce{LiP(TMS)2 + CO(OMe)2 -> LiPCO + 2TMSOMe}
\end{center}
此后,\ce{PCO-}的化学沉寂了将近20年.直到2011年,
Grügtzmacher等人将\ce{NaPH2}与\ce{CO}在$1.0\times10^7\text{ Pa},\ 353\K$下以乙二醇二甲醚(即\ce{DME})为溶剂,反应得到\ce{NaOCP(DME)2},其结构如下所示:
\chemfig{NaOCP(DME)2}{1}{\ce{NaOCP(DME)2}的结构}
\noindent 该反应的方程式为\footnote{在第35届中国化学奥林匹克(决赛)试题中,答案给出的还原产物为\ce{H2},但确实与文献给出的结果不符合.笔者猜测这是命题人为了简化考虑而做出的更改.}:
\begin{center}
    \ce{3NaPH2 + 3CO + DME -> [NaOCP(DME)2]2 + [Na(DME)2][HC(O)PH]}
\end{center}
同年,Cummins等人也用其它方法合成了\ce{NaOCP}:
\begin{center}
    \ce{Na[(Ar^F)_3BP#Nb(NR_1R_2)3] + CO2 -> (Ar^F)_3BONb(NR_1R_2)3 + NaOCP}
\end{center}
除此之外的合成方法在综述中亦有记载,这里就不再展开介绍了.
\paragraph{\ce{PCO-}的反应}
\ce{PCO-}非常容易被氧化.\ce{NaOCP}与\ce{SO2}反应时,\ce{PCO-}会被氧化并形成四聚体\ce{[P4C4O4]^2-},其结构如下所示:
\chemfig{[P4C4O4]2-}{1}{\ce{[P4C4O4]^2-}的结构}
氰酸和异氰酸的三聚是容易的,然而酸化\ce{PCO-}的溶液将得到一些复杂的产物而非\ce{HPCO},\ce{HOCP}或它们的三聚体.制备\ce{P3C3(OH)3}可以采取一些迂回的办法:
\chemfig{P3C3(OH)3}{1}{\ce{P3C3(OH)3}的制备方式}
\paragraph{\ce{PCO-}的配位化学}
显然地,\ce{PCO-}的HOMO轨道为\ce{P}原子的p轨道.这意味着\ce{PCO-}在大多数时候都是以\ce{P}原子为配位原子,并且配位构型也以折线型偏多.\\
\indent 这些配合物的最重要的性质就是容易发生\ce{C-P}键的断裂,形成\ce{P}原子配位的配合物.分解产生的\ce{CO}有时也会与金属中心再次配位.下面是一些例子.
\chemfig{WP(OR)4-}{1}{\ce{[PW(ODipp)4]-}的合成}
\chemfig{Mn2P2(CO)8}{1}{\ce{Mn2P2(CO)8}的合成}
此外还有很多有趣的例子,具体可以参看上面给出的参考文献.
\end{document}