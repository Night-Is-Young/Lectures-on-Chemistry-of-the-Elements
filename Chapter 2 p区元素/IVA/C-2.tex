\documentclass[draft]{ctexart}
\usepackage{EC}
\begin{document}
\section{碳及其化合物}
\subsection{碳的氢化物,卤化物和卤氧化物}
\subsection{碳的硫化物及其衍生物}
\subsubsection{碳的低硫化物}
\paragraph{一硫化碳\ce{CS}}
与\ce{CO}不同,\ce{CS}是一种非常不稳定的物质,几乎只能作为反应中间体存在.它容易与其它VIA族元素和卤素反应生成\ce{CSSe},\ce{CSTe},\ce{CSX2}等物质.
\paragraph{三硫化二碳\ce{C3S2}}
对\ce{CS2}液体放电即可得到红色的液体\ce{C3S2},它像\ce{C3O2}那样也容易在室温下发生缓慢的聚合反应.
\subsubsection{二硫化碳\ce{CS2}}
\ce{C}的最重要的硫化物就是\ce{CS2}.
\paragraph{\ce{CS2}的物理性质}
\ce{CS2}是无色有毒液体,纯的\ce{CS2}有类似\ce{CHCl3}的芳香甜味,但是通常不纯的工业品因为混有其他硫化物(如\ce{COS})而变为微黄色,并且有令人不愉快的烂萝卜味.\\
\indent \ce{CS2}是良好的溶剂,可溶解硫单质或白磷.它本身在水中可溶,但溶解度并不大,$20\tc$时为$2.17\text{ g/L}$.
\paragraph{\ce{CS2}的制备}
过去,\ce{CS2}由硫蒸气和焦炭直接反应得到:
\begin{center}
    \ce{C + 2S ->T[$800\sim1000\tc$] CS2}
\end{center}
现在则主要通过天然气与硫蒸气的反应得到:
\begin{center}
    \ce{CH4 + 4S ->T[$600\tc$][\ce{SiO2/Al2O3}] CS2 + 2H2S}
\end{center}
这一反应类似甲烷在空气中的燃烧.
\paragraph{\ce{CS2}的性质,反应和用途}
在高压下,\ce{CS2}可以聚合形成链状的\ce{(CS2)_n},其中的\ce{C}为平面三角形配位:
\chemfig{(CS2)n}{1}{\ce{CS2}在高压下形成的链状聚合物}
\indent \ce{CS2}虽然在水中的溶解度不大,但可以与碱的溶液反应生成碳酸盐和硫代碳酸盐的混合物:
\begin{center}
    \ce{3CS2 + 6NaOH -> Na2CO3 + 2Na2CS3 + 3H2O}
\end{center}
\ce{CS2}也可以直接溶于含\ce{S^2-}的溶液中形成\ce{[CS3]^2-}:
\begin{center}
    \ce{CS2 + Na2S -> Na2CS3}
\end{center}

\indent 和\ce{CO2}类似,它与\ce{NH3}反应得到双硫代氨基甲酸铵,在\ce{Al2O3}存在时则生成硫氰酸铵\ce{NH4NCS},后者在加热时亦可转化为硫脲\ce{CS(NH2)2}:
\begin{center}
    \ce{CS2 + 2NH3 -> [NH4][H2NCS2]}\\
    \ce{CS2 + 2NH3 ->T[\ce{Al2O3}] NH4SCN + H2S}\\
    \ce{NH4SCN ->T[$160\tc$] CS(NH2)2}
\end{center}
类似地,\ce{CS2}与\ce{HNEt2}在\ce{NaOH}的水溶液中反应可以得到\ce{Et2NCS2Na},它可以作为市售无水乙醚的防爆剂:
\begin{center}
    \ce{NaOH + HNEt2 + CS2 -> Et2NCS2Na + H2O}
\end{center}

\indent \ce{CS2}与\ce{H2O}只能勉强地反应,在$200\tc$生成\ce{H2S}和\ce{COS},在更高温度下则生成\ce{H2S}和\ce{CO2},许多其它含氧化合物都可以与\ce{CS2}反应生成\ce{COS}.\\
\indent \ce{CS2}的乙醇溶液与\ce{NaOH}水溶液反应生成乙基二硫代碳酸钠,即黄原酸钠:
\begin{center}
    \ce{CS2 + NaOH + EtOH -> EtOCS2Na}
\end{center}
将上述反应中的\ce{EtOH}替换为纤维素,就得到黄原酸钠纤维,它溶解在碱的水溶液中得到粘胶纤维,然后重新酸化使纤维素再生即可得到再生纤维素,它是粘液丝,塞珞玢(即我们常说的玻璃纸)的主要成分.这是\ce{CS2}在工业上的主要用途.此外,各种黄原酸盐也可以在选矿过程中作为浮选剂.\\
\indent 用\ce{Na}单质还原\ce{CS2}可以得到具有环状结构的\ce{Na2C3S5},反应的方程式为
\begin{center}
    \ce{4Na + 4CS2 -> Na2C3S5 + Na2CS3}
\end{center}
\ce{[C3S5]^2-}的结构如下所示:
\chemfig{C3S52-}{1}{\ce{[C3S5]^2-}的结构}
\indent \ce{CS2}的氯化可以得到\ce{CCl4}.由\ce{Fe/FeCl3}催化时,反应分为两步进行:
\begin{center}
    \ce{CS2 + 3Cl2 ->T[\ce{Fe/FeCl3}] CCl4 + S2Cl2}\\
    \ce{CS2 + 2S2Cl2 ->T[\ce{Fe/FeCl3}] CCl4 + 6S}
\end{center}
1843-1845年,H. Kolbe以这一反应为第一步,以\ce{CS2}为原料之一完成了乙酸的合成,证明有机物可以从无机物人工制得,否定了生命力学说.\\
\indent 在圆柱形的管中点燃\ce{CS2}与\ce{N2O}的混合物,会发出明亮的蓝色闪光和类似吠叫的响声(因此这一反应也被称为\tbf{狗吠反应}),反应的方程式为
\begin{center}
    \ce{8N2O + 4CS2 -> S8 + 4CO2 + 8N2}
\end{center}
\subsubsection{羰基硫\ce{COS}}
\paragraph{\ce{COS}的物理性质}
\ce{COS}是无色,有臭鸡蛋味的有毒气体\footnote{它和\ce{H2S}一样容易让人对其浓度产生低估,这进一步增加了其危险性.},可燃.
\paragraph{\ce{COS}的制备}
\ce{COS}可以由\ce{CS2}的部分水解得到:
\begin{center}
    \ce{CS2 + H2O -> COS + H2S}
\end{center}
在制备\ce{CS2}的过程中通常也混有少量的\ce{COS}.\\
\indent \ce{COS}可以通过\ce{CO}与硫单质的反应得到:
\begin{center}
    \ce{CO + S -> COS}
\end{center}

\indent 实验室中可以通过硫氰酸盐与浓硫酸的反应制取\ce{COS}:
\begin{center}
    \ce{KCSN + 2H2SO4 + H2O -> KHSO4 + NH4HSO4 + COS}
\end{center}
反应通常产生大量的副产物,因此要经冷却提纯才能得到比较纯净的\ce{COS}.
\subsection{碳的氮化物及其衍生物}
\subsubsection{石墨氮化碳}
将石墨层的结构稍加修改,就能得到化学式为\ce{C3N4}的石墨氮化碳.和石墨一样,\ce{C3N4}也具有无限延伸的二维层状结构,并且有以下两种异构体:
\bichemfig{C3N4-1}{0.9}{\ce{C3N4}的层状结构I}{C3N4-2}{0.9}{\ce{C3N4}的层状结构II}{\ce{C3N4}的层状结构}
如果你仔细观察,就可以发现这些层状结构是由$\dfrac{n(n+1)}{2}$个三聚吡啶并接成的边长为$n$的正三角形单元和连接这些单元的\ce{N}原子组成的.然而,当$n>2$时,由于成键的限制将不能画出类似的结构.因此\ce{C3N4}似乎仅有上面两种比较合理的异构体.\\
\indent 石墨氮化碳作为一种新型二维材料,在催化和能源等领域有着重要用途.
\subsubsection{氰及其衍生物}
\paragraph{关于氰的基本介绍}
所谓氰,除去指氰气外,通常而言指\ce{-CN}基团或\ce{CN-}离子.它得名于形成深蓝色色素的性质,例如与铁盐形成普鲁士蓝,其希腊语为$\kappa\acute{\nu}\alpha\eta o\varsigma$,对应的英文为$cyanos$,意为“深蓝”.\\
\indent 与氰相关的另一个重要的概念是\tbf{拟卤素}.通常,类似\ce{CN,OCN,SCN}等含有氰基的基团在一般的反应中不会发生改变,并且和卤素一样是一价的.这些基团能形成阴离子\ce{X-},氢化物(通常是酸)\ce{HX},有时形成中性分子\ce{X2},\ce{XY}等等.种种性质表明它们与卤素有一定的类似,因此称它们为拟卤素.\\
\indent 此外,可以发现\ce{CN-}和\ce{CO},\ce{N2}和\ce{NO+}是等电子体.与此类似的,\ce{OCN-}与\ce{CO2},\ce{N3^-}是等电子体.\\
\indent 现在,我们介绍氰及其衍生物中主要的几种.
\paragraph{氰\ce{(CN)2}}
氰是碳的最简单的氮化物,由两个\ce{CN}基团连接而成.
\subparagraph{\ce{(CN)2}的物理性质}
\ce{(CN)2}是无色的苦杏仁味气体,有剧毒.\ce{(CN)2}可溶于水,乙醇,乙醚.\\
\indent \ce{(CN)2}可燃,燃烧时呈桃红色火焰,边缘侧带蓝色.\ce{(CN)2}在纯氧中的燃烧可以达到$4525\tc$以上的高温,仅次于二氰乙炔\ce{C2(CN)2}.
\subparagraph{\ce{(CN)2}的制备}
实验室中通常可以采取\ce{KCN}与\ce{CuSO4}的反应制取\ce{(CN)2}.这可以类比\ce{Cu^2+}与\ce{I-}的反应,它们都生成了卤素/拟卤素“单质”和\ce{Cu^I}的难溶盐:
\begin{center}
    \ce{2CuSO4 + 4KCN -> (CN)2 + 2CuCN + 2K2SO4}
\end{center}
该反应的产率在$80\%$左右,主要的副产物为\ce{CO2}.副产物\ce{CuCN}可以进一步用\ce{FeCl3}的热水溶液氧化以释放其中的\ce{CN-}:
\begin{center}
    \ce{2CuCN + 2FeCl3 ->T[$\Delta$] (CN)2 + 2CuCl + 2FeCl2}
\end{center}

\indent 工业上主要采取氰化氢的催化氧化制得:
\begin{center}
    \ce{4HCN + O2 ->T[\ce{Ag}] 2H2O + 2(CN)2}\\
    \ce{2HCN + Cl2 ->T[\ce{CaO/SiO2}] 2HCl + (CN)2}
\end{center}

\indent 此外,对草酰胺脱水也得到\ce{(CN)2}.不过这不是主要的制备方法,但可以据此而认为氰是草酸衍生的腈.
\begin{center}
    \ce{H2NCOCONH2 ->T[脱水] (CN)2 + 2H2O}
\end{center}
\subparagraph{\ce{(CN)2}的性质与反应}
\ce{(CN)2}在一般的溶剂中比较稳定,但在碱性水溶液中发生类似卤素单质的歧化,生成氰根离子\ce{CN-}和氧氰阴离子\ce{OCN-}:
\begin{center}
    \ce{(CN)2 + 2OH- -> CN- + OCN- + H2O}
\end{center}
\end{document}