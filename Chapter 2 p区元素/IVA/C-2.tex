\documentclass{ctexart}
\usepackage{EC}
\begin{document}
\section{碳及其化合物}
\subsection{碳的氢化物,卤化物和卤氧化物}
\subsection{碳的硫化物及其衍生物}
\subsubsection{碳的低硫化物}
\paragraph{一硫化碳\ce{CS}}
与\ce{CO}不同,\ce{CS}是一种非常不稳定的物质,几乎只能作为反应中间体存在.它容易与其它VIA族元素和卤素反应生成\ce{CSSe},\ce{CSTe},\ce{CSX2}等物质.
\paragraph{三硫化二碳\ce{C3S2}}
对\ce{CS2}液体放电即可得到红色的液体\ce{C3S2},它像\ce{C3O2}那样也容易在室温下发生缓慢的聚合反应.
\subsubsection{二硫化碳\ce{CS2}}
\ce{C}的最重要的硫化物就是\ce{CS2}.
\paragraph{\ce{CS2}的物理性质}
\ce{CS2}是无色有毒液体,纯的\ce{CS2}有类似\ce{CHCl3}的芳香甜味,但是通常不纯的工业品因为混有其他硫化物(如\ce{COS})而变为微黄色,并且有令人不愉快的烂萝卜味.\\
\indent \ce{CS2}是良好的溶剂,可溶解硫单质或白磷.它本身在水中可溶,但溶解度并不大,$20\tc$时为$2.17\text{ g/L}$.
\paragraph{\ce{CS2}的制备}
过去,\ce{CS2}由硫蒸气和焦炭直接反应得到:
\begin{center}
    \ce{C + 2S ->T[$800\sim1000\tc$] CS2}
\end{center}
现在则主要通过天然气与硫蒸气的反应得到:
\begin{center}
    \ce{CH4 + 4S ->T[$600\tc$][\ce{SiO2/Al2O3}] CS2 + 2H2S}
\end{center}
这一反应类似甲烷在空气中的燃烧.
\paragraph{\ce{CS2}的性质,反应和用途}
在高压下,\ce{CS2}可以聚合形成链状的\ce{(CS2)_n},其中的\ce{C}为平面三角形配位:
\chemfig{(CS2)n}{1}{\ce{CS2}在高压下形成的链状聚合物}
\indent \ce{CS2}虽然在水中的溶解度不大,但可以与碱的溶液反应生成碳酸盐和硫代碳酸盐的混合物:
\begin{center}
    \ce{3CS2 + 6NaOH -> Na2CO3 + 2Na2CS3 + 3H2O}
\end{center}
\ce{CS2}也可以直接溶于含\ce{S^2-}的溶液中形成\ce{[CS3]^2-}:
\begin{center}
    \ce{CS2 + Na2S -> Na2CS3}
\end{center}

\indent 和\ce{CO2}类似,它与\ce{NH3}反应得到双硫代氨基甲酸铵,在\ce{Al2O3}存在时则生成硫氰酸铵\ce{NH4NCS},后者在加热时亦可转化为硫脲\ce{CS(NH2)2}:
\begin{center}
    \ce{CS2 + 2NH3 -> [NH4][H2NCS2]}\\
    \ce{CS2 + 2NH3 ->T[\ce{Al2O3}] NH4SCN + H2S}\\
    \ce{NH4SCN ->T[$160\tc$] CS(NH2)2}
\end{center}
类似地,\ce{CS2}与\ce{HNEt2}在\ce{NaOH}的水溶液中反应可以得到\ce{Et2NCS2Na},它可以作为市售无水乙醚的防爆剂:
\begin{center}
    \ce{NaOH + HNEt2 + CS2 -> Et2NCS2Na + H2O}
\end{center}

\indent \ce{CS2}与\ce{H2O}只能勉强地反应,在$200\tc$生成\ce{H2S}和\ce{COS},在更高温度下则生成\ce{H2S}和\ce{CO2},许多其它含氧化合物都可以与\ce{CS2}反应生成\ce{COS}.\\
\indent \ce{CS2}的乙醇溶液与\ce{NaOH}水溶液反应生成乙基二硫代碳酸钠,即黄原酸钠:
\begin{center}
    \ce{CS2 + NaOH + EtOH -> EtOCS2Na}
\end{center}
将上述反应中的\ce{EtOH}替换为纤维素,就得到黄原酸钠纤维,它溶解在碱的水溶液中得到粘胶纤维,然后重新酸化使纤维素再生即可得到再生纤维素,它是粘液丝,塞珞玢(即我们常说的玻璃纸)的主要成分.这是\ce{CS2}在工业上的主要用途.此外,各种黄原酸盐也可以在选矿过程中作为浮选剂.\\
\indent 用\ce{Na}单质还原\ce{CS2}可以得到具有环状结构的\ce{Na2C3S5},反应的方程式为
\begin{center}
    \ce{4Na + 4CS2 -> Na2C3S5 + Na2CS3}
\end{center}
\ce{[C3S5]^2-}的结构如下所示:
\chemfig{C3S52-}{1}{\ce{[C3S5]^2-}的结构}
\indent \ce{CS2}的氯化可以得到\ce{CCl4}.由\ce{Fe/FeCl3}催化时,反应分为两步进行:
\begin{center}
    \ce{CS2 + 3Cl2 ->T[\ce{Fe/FeCl3}] CCl4 + S2Cl2}\\
    \ce{CS2 + 2S2Cl2 ->T[\ce{Fe/FeCl3}] CCl4 + 6S}
\end{center}
1843-1845年,H. Kolbe以这一反应为第一步,以\ce{CS2}为原料之一完成了乙酸的合成,证明有机物可以从无机物人工制得,否定了生命力学说.\\
\indent 在圆柱形的管中点燃\ce{CS2}与\ce{N2O}的混合物,会发出明亮的蓝色闪光和类似吠叫的响声(因此这一反应也被称为\tbf{狗吠反应}),反应的方程式为
\begin{center}
    \ce{8N2O + 4CS2 -> S8 + 4CO2 + 8N2}
\end{center}
\subsubsection{羰基硫\ce{COS}}
\paragraph{\ce{COS}的物理性质}
\ce{COS}是无色,有臭鸡蛋味的有毒气体\footnote{它和\ce{H2S}一样容易让人对其浓度产生低估,这进一步增加了其危险性.},可燃.
\paragraph{\ce{COS}的制备}
\ce{COS}可以由\ce{CS2}的部分水解得到:
\begin{center}
    \ce{CS2 + H2O -> COS + H2S}
\end{center}
在制备\ce{CS2}的过程中通常也混有少量的\ce{COS}.\\
\indent \ce{COS}可以通过\ce{CO}与硫单质的反应得到:
\begin{center}
    \ce{CO + S -> COS}
\end{center}

\indent 实验室中可以通过硫氰酸盐与浓硫酸的反应制取\ce{COS}:
\begin{center}
    \ce{KCSN + 2H2SO4 + H2O -> KHSO4 + NH4HSO4 + COS}
\end{center}
反应通常产生大量的副产物,因此要经冷却提纯才能得到比较纯净的\ce{COS}.
\subsection{碳的氮化物及其衍生物}
\subsubsection{石墨氮化碳}
将石墨层的结构稍加修改,就能得到化学式为\ce{C3N4}的石墨氮化碳.和石墨一样,\ce{C3N4}也具有无限延伸的二维层状结构,并且有以下两种异构体:
\bichemfig{C3N4-1}{0.9}{\ce{C3N4}的层状结构I}{C3N4-2}{0.9}{\ce{C3N4}的层状结构II}{\ce{C3N4}的层状结构}
如果你仔细观察,就可以发现这些层状结构是由$\dfrac{n(n+1)}{2}$个均三嗪并接成的边长为$n$的正三角形单元和连接这些单元的\ce{N}原子组成的.然而,当$n>2$时,由于成键的限制将不能画出类似的结构.因此\ce{C3N4}似乎仅有上面两种比较合理的异构体.\\
\indent 石墨氮化碳作为一种新型二维材料,在催化和能源等领域有着重要用途.
\subsubsection{氰及其衍生物}
\paragraph{关于氰的基本介绍}
所谓氰,除去指氰气外,通常而言指\ce{-CN}基团或\ce{CN-}离子.它得名于形成深蓝色色素的性质,例如与铁盐形成普鲁士蓝,其希腊语为$\kappa\acute{\nu}\alpha\eta o\varsigma$,对应的英文为$cyanos$,意为“深蓝”.\\
\indent 与氰相关的另一个重要的概念是\tbf{拟卤素}.通常,类似\ce{CN,OCN,SCN}等含有氰基的基团在一般的反应中不会发生改变,并且和卤素一样是一价的.这些基团能形成阴离子\ce{X-},氢化物(通常是酸)\ce{HX},有时形成中性分子\ce{X2},\ce{XY}等等.种种性质表明它们与卤素有一定的类似,因此称它们为拟卤素.\\
\indent 此外,可以发现\ce{CN-}和\ce{CO},\ce{N2}和\ce{NO+}是等电子体.与此类似的,\ce{OCN-}与\ce{CO2},\ce{N3^-}是等电子体.\\
\indent 现在,我们介绍氰及其衍生物中主要的几种.
\paragraph{氰\ce{(CN)2}}
氰是碳的最简单的氮化物,由两个\ce{CN}基团连接而成.
\subparagraph{\ce{(CN)2}的物理性质}
\ce{(CN)2}是无色的苦杏仁味气体,有剧毒.\ce{(CN)2}可溶于水,乙醇,乙醚.\\
\indent \ce{(CN)2}可燃,燃烧时呈桃红色火焰,边缘侧带蓝色.\ce{(CN)2}在纯氧中的燃烧可以达到$4525\tc$以上的高温,仅次于二氰乙炔\ce{C2(CN)2}.
\subparagraph{\ce{(CN)2}的制备}
实验室中通常可以采取\ce{KCN}与\ce{CuSO4}的反应制取\ce{(CN)2}.这可以类比\ce{Cu^2+}与\ce{I-}的反应,它们都生成了卤素/拟卤素“单质”和\ce{Cu^I}的难溶盐:
\begin{center}
    \ce{2CuSO4 + 4KCN -> (CN)2 + 2CuCN + 2K2SO4}
\end{center}
该反应的产率在$80\%$左右,主要的副产物为\ce{CO2}.副产物\ce{CuCN}可以进一步用\ce{FeCl3}的热水溶液氧化以释放其中的\ce{CN-}:
\begin{center}
    \ce{2CuCN + 2FeCl3 ->T[$\Delta$] (CN)2 + 2CuCl + 2FeCl2}
\end{center}

\indent 工业上主要采取氰化氢的催化氧化制得:
\begin{center}
    \ce{4HCN + O2 ->T[\ce{Ag}] 2H2O + 2(CN)2}\\
    \ce{2HCN + Cl2 ->T[\ce{CaO/SiO2}] 2HCl + (CN)2}
\end{center}

\indent 此外,对草酰胺脱水也得到\ce{(CN)2}.不过这不是主要的制备方法,但可以据此而认为氰是草酸衍生的腈.
\begin{center}
    \ce{H2NCOCONH2 ->T[脱水] (CN)2 + 2H2O}
\end{center}
\subparagraph{\ce{(CN)2}的性质与反应}
纯的\ce{(CN)2}具有相当高的热稳定性,但痕量杂质的存在将使得其在$300\sim500\tc$时发生聚合而生成顺氰,其中具有吡嗪环并接的长链结构:
\chemfig{(CN)n}{1}{顺氰的结构}
\ce{(CN)2}在一般的溶剂中比较稳定,但在碱性水溶液中发生类似卤素单质的歧化,生成氰根离子\ce{CN-}和氧氰阴离子\ce{OCN-}:
\begin{center}
    \ce{(CN)2 + 2OH- -> CN- + OCN- + H2O}
\end{center}
\paragraph{氰化氢\ce{HCN}}
\subparagraph{\ce{HCN}的物理性质}
\ce{HCN}是无色的杏仁味气体\footnote{这种气味主要是因为杏仁核中的苦杏仁苷水解所产生的\ce{HCN}.此事在Clayden等所著的有机化学书中有详细的记载.}\footnote{据说能否明显地闻到此气味却决于个人的基因.},剧毒且致命.\ce{HCN}的熔点为$-13.4\tc$,沸点为$25.6\tc$,因而在标准状态下是液体.
\subparagraph{\ce{HCN}的制备}
以前,制备\ce{HCN}主要靠\ce{NaCN}和\ce{Ca(CN)2}等氰化物盐的酸化.甲酸铵的脱水也可以得到\ce{HCN}:
\begin{center}
    \ce{HCOONH4 -> HCN + 2H2O}
\end{center}
现在通常通过催化反应合成\ce{HCN}:
\begin{center}
    \tbf{Andrussow过程}:\ \ce{2CH4 + 2NH3 + 3O2 ->T[\ce{Pt/Rh/Ir}][$\Delta$] 2HCN + 6H2O}\\
    \tbf{Degussa过程}:\ \ce{CH4 + NH3 ->T[\ce{Pt}][$\Delta$] HCN + 3H2}
\end{center}
\subparagraph{\ce{HCN}的性质,反应与应用}
\ce{HCN}在水中是弱酸:
\begin{center}
    \ce{HCN(aq) + H2O(l) <=> H3O+(aq) + CN-(aq)}\ \ \ $K_{\text a}=2.1\times10^{-9}$
\end{center}
它也是中等的还原剂.
\begin{center}
    \ce{(CN)2 + 2H+ + 2e- -> 2HCN}\ \ \ $E^\ominus=+0.37\text{ V}$
\end{center}
用较强的还原剂,例如\ce{Pd/H2}即可将\ce{HCN}还原为\ce{MeNH2}.\\
\indent 液态\ce{HCN}容易发生聚合.三聚体和四聚体的结构如下所示.
\bichemfig{(HCN)3}{1}{\ce{HCN}的三聚体结构}{(HCN)4}{1}{\ce{HCN}的四聚体的结构}{\ce{HCN}的多聚体结构}
\indent 其它的性质主要是\ce{CN-}的性质.因此,我们放到下一小节介绍.
\paragraph{氰根阴离子\ce{CN-}与氰化物}
\subparagraph{\ce{CN-}的毒性}
应当说,绝大部分氰类化合物的毒性都来源于\ce{CN-}.它与细胞色素c氧化酶中的\ce{Fe}原子结合,抑制其活性,从而干扰正常有氧呼吸的过程.高度依赖有氧呼吸的组织,例如心脏和中枢神经系统,受此影响最大.\\
\indent 氰化物中毒的临床症状包括:中毒者血液的pH值在中毒后两至三分钟内急剧下降\footnote{这应当是因为细胞只能采取无氧呼吸,产生大量丙酮酸的缘故.};缺氧窒息;身体散发大量类似苦杏仁味的气味;严重昏迷及面部发紫;即使痊愈后,大部分中毒者的脑部和心脏一般都已受永久性伤害.\\
\indent 在众多侦探和悬疑小说中,通常用于毒杀他人的物质就是氢氰酸.综合而言,它容易得到,并且毒性足够强烈.现在,氰化物都列入严格管制,以免不法分子用它危害他人生命.
\subparagraph{氰化物的制备}
主要需要制备的简单氰化物通常是碱金属和碱土金属的氰化物.\ce{NaCN}的早期的工业制备方法是将\ce{NaNH2}与碳反应得到:
\begin{center}
    \ce{NaNH2 + C ->T[$\Delta$] NaCN + H2}
\end{center}
现在则由氰氨化钙,碳和碳酸钠反应得到:
\begin{center}
    \ce{CaCN2 + Na2CO3 + C -> 2NaCN + CaCO3}
\end{center}
\ce{NaCN}大量地用于配制金属电镀液.
\subparagraph{\ce{CN-}的配位化学与反应}
和它的等电子体\ce{CO}一样,\ce{CN-}也有着十分强的配位能力,其配合物也数目众多,并且一般而言用\ce{C}作为配位原子.我们在这里简单地对\ce{CN-}的配合物进行介绍,剩下的内容放到各自对应的中心元素中.\\
\indent 氰化物曾经被大量用于金和银的开采中,它有助于溶解这些金属,从而与其他固体分离.反应的方程式如下:
\begin{center}
    \ce{4Au + 8NaCN + O2 + 2H2O -> 4Na[Au(CN)2] + 4NaOH}\\
    \ce{Ag2S + 4NaCN + H2O -> 2Na[Ag(CN)2] + NaSH + NaOH}
\end{center}
其中\ce{[Au(CN)2]-}和\ce{[Ag(CN)2]-}都是直线形的阴离子.由于氰化物的剧毒性,这种方法对环境的破坏是相当大的,因此已经被禁止使用.然而在某些手工提取电子产品中的金的视频中,似乎仍然有人采用这种不规范的方法.\\
\indent 除此之外,\ce{CN-}也能作为桥连配体.在\ce{AgCN}和\ce{AuCN}中,每个\ce{CN-}分别用\ce{C}和\ce{N}对两个\ce{M+}配位,形成无限延伸的一维长链.
\bichemfig{M(CN)2-}{1}{\ce{[M(CN)2]-}的结构}{MCN}{1}{\ce{MCN}中的一维长链}{$\ce{M}(\ce{M}=\ce{Ag},\ce{Au})$的\ce{CN-}配合物}
这种桥连的形式还存在于很多物质中,例如\ce{[Co2(NH3)10(CN)]^5+},\ce{[Au(C3H7)2(CN)]4}和层型结构的\ce{Pd(CN)2}等.这些物质的结构如下所示.
\chemfig{Co2(NH3)10(CN)5+}{1}{\ce{[Co2(NH3)10(CN)]^5+}的结构}
\bichemfig{(AuR2CN)4}{1}{\ce{[Au(C3H7)2(CN)]4}}{Pd(CN)2}{1}{\ce{Pd(CN)2}的结构}{\ce{CN-}桥连的平面四方配合物}
在\ce{CuCN.NH3}中有\ce{CN-}桥连的二维结构(除去对\ce{Cu}配位的\ce{NH3}),但与上面不同的是\ce{C}同时对两个\ce{Cu^I}配位,同时这两个\ce{Cu^I}之间还有类似亲金作用的亲铜作用.这一层状结构如下所示.
\chemfig{CuCN}{1}{\ce{CuCN.NH3}的二维层}
\paragraph{氰的卤化物\ce{XCN}}
氰与四种卤素的化合物都已经制得.\ce{ClCN}和\ce{BrCN}可以由对应的卤素与碱金属氰化物反应得到:
\begin{center}
    \ce{X2 + MCN -> MCl + XCN}
\end{center}
将干燥的\ce{Hg(CN)2}与\ce{I2}反应即可得到\ce{ICN}:
\begin{center}
    \ce{Hg(CN)2 + 2I2 -> 2ICN + HgI2}
\end{center}
而\ce{FCN}则由\ce{(FCN)3}的分解得到.后者可以由\ce{NaF}与\ce{(ClCN)3}反应得到.事实上,所有四种\ce{XCN}都可以发生三聚形成含有六元环的\ce{(XCN)3},其结构如下:
\chemfig{(XCN)3}{1}{\ce{(XCN)3}的结构}
\ce{XCN}可以发生取代反应,得到各种含有氰基的化合物.以下是一些例子:
\begin{center}
    \ce{XCN + NaN3 -> N3CN + NaX}\\
    \ce{XCN + 2NaOH + NaX + NaOCN + H2O}\\
    \ce{ClCN + AgOCN -> NCOCN + AgCl}
\end{center}
\paragraph{氨基氰\ce{H2NCN}及其衍生物}
\subparagraph{氨基氰的制备与结构}
氨基氰(即氰胺)可以由氰氨化钙按如下方式制备得到:
\begin{center}
    \ce{2CaCN2 + 2H2O -> Ca(HNCN)2 + Ca(OH)2}\\
    \ce{Ca(HNCN)2 + CO2 + H2O -> CaCO3 + 2H2NCN}
\end{center}
氰胺是平面型分子,其结构如下:
\chemfig{H2NCN}{1}{氰胺分子的结构}
氰胺也有一个不稳定的异构体,即二亚胺\ce{HN=C=NH}.有机化学中常用的脱水剂DCC(即二环己基二亚胺)就是二亚胺的衍生物.
\subparagraph{氨基氰的反应}
\indent 常用的氰胺是以二聚体的形式存在的.这二聚体也有一个异构体,如下所示.
\bichemfig{(H2NCN)2-1}{1}{\ce{H2NCN}的二聚体I}{(H2NCN)2-2}{1}{\ce{H2NCN}的二聚体II}{\ce{H2NCN}的二聚体}
氰胺可以发生三聚.三聚氰胺常用作制备树脂,用作阻燃剂,减水剂和甲醛清洁剂等等.不法商家曾经将三聚氰胺加入奶粉中以提高氮含量,进而让检测出的蛋白质含量更高\footnote{这就是曾经轰动全国的著名的三鹿奶粉事件.}.由于三聚氰胺对神经系统的毒害作用,这对婴幼儿有极其恶劣的健康影响.
\chemfig{(H2NCN)3}{1}{三聚氰胺的结构}

\end{document}