\documentclass{ctexart}
\usepackage{EC}
\begin{document}
\section{碳及其化合物}
\subsection{碳的氢化物,卤化物和卤氧化物}
\subsection{碳的氮化物及其衍生物}
\subsubsection{石墨氮化碳}
将石墨层的结构稍加修改,就能得到化学式为\ce{C3N4}的石墨氮化碳.和石墨一样,\ce{C3N4}也具有无限延伸的二维层状结构,并且有以下两种异构体:
\bichemfig{C3N4-1}{0.9}{\ce{C3N4}的层状结构I}{C3N4-2}{0.9}{\ce{C3N4}的层状结构II}{\ce{C3N4}的层状结构}
如果你仔细观察,就可以发现这些层状结构是由$\dfrac{n(n+1)}{2}$个均三嗪并接成的边长为$n$的正三角形单元和连接这些单元的\ce{N}原子组成的.然而,当$n>2$时,由于成键的限制将不能画出类似的结构.因此\ce{C3N4}似乎仅有上面两种比较合理的异构体.\\
\indent 石墨氮化碳作为一种新型二维材料,在催化和能源等领域有着重要用途.
\subsubsection{氰及其衍生物}
\paragraph{关于氰的基本介绍}
所谓氰,除去指氰气外,通常而言指\ce{-CN}基团或\ce{CN-}离子.它得名于形成深蓝色色素的性质,例如与铁盐形成普鲁士蓝,其希腊语为$\kappa\acute{\nu}\alpha\eta o\varsigma$,对应的英文为$cyanos$,意为“深蓝”.\\
\indent 与氰相关的另一个重要的概念是\tbf{拟卤素}.通常,类似\ce{CN,OCN,SCN}等含有氰基的基团在一般的反应中不会发生改变,并且和卤素一样是一价的.这些基团能形成阴离子\ce{X-},氢化物(通常是酸)\ce{HX},有时形成中性分子\ce{X2},\ce{XY}等等.种种性质表明它们与卤素有一定的类似,因此称它们为拟卤素.\\
\indent 此外,可以发现\ce{CN-}和\ce{CO},\ce{N2}和\ce{NO+}是等电子体.与此类似的,\ce{OCN-}与\ce{CO2},\ce{N3^-}是等电子体.\\
\indent 现在,我们介绍氰及其衍生物中主要的几种.
\paragraph{氰\ce{(CN)2}}
氰是碳的最简单的氮化物,由两个\ce{CN}基团连接而成.
\subparagraph{\ce{(CN)2}的物理性质}
\ce{(CN)2}是无色的苦杏仁味气体,有剧毒.\ce{(CN)2}可溶于水,乙醇,乙醚.\\
\indent \ce{(CN)2}可燃,燃烧时呈桃红色火焰,边缘侧带蓝色.\ce{(CN)2}在纯氧中的燃烧可以达到$4525\tc$以上的高温,仅次于二氰乙炔\ce{C2(CN)2}.
\subparagraph{\ce{(CN)2}的制备}
实验室中通常可以采取\ce{KCN}与\ce{CuSO4}的反应制取\ce{(CN)2}.这可以类比\ce{Cu^2+}与\ce{I-}的反应,它们都生成了卤素/拟卤素“单质”和\ce{Cu^I}的难溶盐:
\begin{center}
    \ce{2CuSO4 + 4KCN -> (CN)2 + 2CuCN + 2K2SO4}
\end{center}
该反应的产率在$80\%$左右,主要的副产物为\ce{CO2}.副产物\ce{CuCN}可以进一步用\ce{FeCl3}的热水溶液氧化以释放其中的\ce{CN-}:
\begin{center}
    \ce{2CuCN + 2FeCl3 ->T[$\Delta$] (CN)2 + 2CuCl + 2FeCl2}
\end{center}

\indent 工业上主要采取氰化氢的催化氧化制得:
\begin{center}
    \ce{4HCN + O2 ->T[\ce{Ag}] 2H2O + 2(CN)2}\\
    \ce{2HCN + Cl2 ->T[\ce{CaO/SiO2}] 2HCl + (CN)2}
\end{center}

\indent 此外,对草酰胺脱水也得到\ce{(CN)2}.不过这不是主要的制备方法,但可以据此而认为氰是草酸衍生的腈.
\begin{center}
    \ce{H2NCOCONH2 ->T[脱水] (CN)2 + 2H2O}
\end{center}
\subparagraph{\ce{(CN)2}的性质与反应}
纯的\ce{(CN)2}具有相当高的热稳定性,但痕量杂质的存在将使得其在$300\sim500\tc$时发生聚合而生成顺氰,其中具有吡嗪环并接的长链结构:
\chemfig{(CN)n}{1}{顺氰的结构}
\ce{(CN)2}在一般的溶剂中比较稳定,但在碱性水溶液中发生类似卤素单质的歧化,生成氰根离子\ce{CN-}和氧氰阴离子\ce{OCN-}:
\begin{center}
    \ce{(CN)2 + 2OH- -> CN- + OCN- + H2O}
\end{center}
\paragraph{氰化氢\ce{HCN}}
\subparagraph{\ce{HCN}的物理性质}
\ce{HCN}是无色的杏仁味气体\footnote{这种气味主要是因为杏仁核中的苦杏仁苷水解所产生的\ce{HCN}.此事在Clayden等所著的有机化学书中有详细的记载.}\footnote{据说能否明显地闻到此气味却决于个人的基因.},剧毒且致命.\ce{HCN}的熔点为$-13.4\tc$,沸点为$25.6\tc$,因而在标准状态下是液体.
\subparagraph{\ce{HCN}的制备}
以前,制备\ce{HCN}主要靠\ce{NaCN}和\ce{Ca(CN)2}等氰化物盐的酸化.甲酸铵的脱水也可以得到\ce{HCN}:
\begin{center}
    \ce{HCOONH4 -> HCN + 2H2O}
\end{center}
现在通常通过催化反应合成\ce{HCN}:
\begin{center}
    \tbf{Andrussow过程}:\ \ce{2CH4 + 2NH3 + 3O2 ->T[\ce{Pt/Rh/Ir}][$\Delta$] 2HCN + 6H2O}\\
    \tbf{Degussa过程}:\ \ce{CH4 + NH3 ->T[\ce{Pt}][$\Delta$] HCN + 3H2}
\end{center}
\subparagraph{\ce{HCN}的性质,反应与应用}
\ce{HCN}在水中是弱酸:
\begin{center}
    \ce{HCN(aq) + H2O(l) <=> H3O+(aq) + CN-(aq)}\ \ \ $K_{\text a}=2.1\times10^{-9}$
\end{center}
它也是中等的还原剂.
\begin{center}
    \ce{(CN)2 + 2H+ + 2e- -> 2HCN}\ \ \ $E^\ominus=+0.37\text{ V}$
\end{center}
用较强的还原剂,例如\ce{Pd/H2}即可将\ce{HCN}还原为\ce{MeNH2}.\\
\indent 液态\ce{HCN}容易发生聚合.三聚体和四聚体的结构如下所示.
\bichemfig{(HCN)3}{1}{\ce{HCN}的三聚体结构}{(HCN)4}{1}{\ce{HCN}的四聚体的结构}{\ce{HCN}的多聚体结构}
\indent 其它的性质主要是\ce{CN-}的性质.因此,我们放到下一小节介绍.
\paragraph{氰根阴离子\ce{CN-}与氰化物}
\subparagraph{\ce{CN-}的毒性}
应当说,绝大部分氰类化合物的毒性都来源于\ce{CN-}.它与细胞色素c氧化酶中的\ce{Fe}原子结合,抑制其活性,从而干扰正常有氧呼吸的过程.高度依赖有氧呼吸的组织,例如心脏和中枢神经系统,受此影响最大.\\
\indent 氰化物中毒的临床症状包括:中毒者血液的pH值在中毒后两至三分钟内急剧下降\footnote{这应当是因为细胞只能采取无氧呼吸,产生大量丙酮酸的缘故.};缺氧窒息;身体散发大量类似苦杏仁味的气味;严重昏迷及面部发紫;即使痊愈后,大部分中毒者的脑部和心脏一般都已受永久性伤害.\\
\indent 在众多侦探和悬疑小说中,通常用于毒杀他人的物质就是氢氰酸.综合而言,它容易得到,并且毒性足够强烈.现在,氰化物都列入严格管制,以免不法分子用它危害他人生命.
\subparagraph{氰化物的制备}
主要需要制备的简单氰化物通常是碱金属和碱土金属的氰化物.\ce{NaCN}的早期的工业制备方法是将\ce{NaNH2}与碳反应得到:
\begin{center}
    \ce{NaNH2 + C ->T[$\Delta$] NaCN + H2}
\end{center}
现在则由氰氨化钙,碳和碳酸钠反应得到:
\begin{center}
    \ce{CaCN2 + Na2CO3 + C -> 2NaCN + CaCO3}
\end{center}
\ce{NaCN}大量地用于配制金属电镀液.
\subparagraph{\ce{CN-}的配位化学与反应}
和它的等电子体\ce{CO}一样,\ce{CN-}也有着十分强的配位能力,其配合物也数目众多,并且一般而言用\ce{C}作为配位原子.我们在这里简单地对\ce{CN-}的配合物进行介绍,剩下的内容放到各自对应的中心元素中.\\
\indent 氰化物曾经被大量用于金和银的开采中,它有助于溶解这些金属,从而与其他固体分离.反应的方程式如下:
\begin{center}
    \ce{4Au + 8NaCN + O2 + 2H2O -> 4Na[Au(CN)2] + 4NaOH}\\
    \ce{Ag2S + 4NaCN + H2O -> 2Na[Ag(CN)2] + NaSH + NaOH}
\end{center}
其中\ce{[Au(CN)2]-}和\ce{[Ag(CN)2]-}都是直线形的阴离子.由于氰化物的剧毒性,这种方法对环境的破坏是相当大的,因此已经被禁止使用.然而在某些手工提取电子产品中的金的视频中,似乎仍然有人采用这种不规范的方法.\\
\indent 除此之外,\ce{CN-}也能作为桥连配体.在\ce{AgCN}和\ce{AuCN}中,每个\ce{CN-}分别用\ce{C}和\ce{N}对两个\ce{M+}配位,形成无限延伸的一维长链.
\bichemfig{M(CN)2-}{1}{\ce{[M(CN)2]-}的结构}{MCN}{1}{\ce{MCN}中的一维长链}{$\ce{M}(\ce{M}=\ce{Ag},\ce{Au})$的\ce{CN-}配合物}
这种桥连的形式还存在于很多物质中,例如\ce{[Co2(NH3)10(CN)]^5+},\ce{[Au(C3H7)2(CN)]4}和层型结构的\ce{Pd(CN)2}等.这些物质的结构如下所示.
\chemfig{Co2(NH3)10(CN)5+}{1}{\ce{[Co2(NH3)10(CN)]^5+}的结构}
\bichemfig{(AuR2CN)4}{1}{\ce{[Au(C3H7)2(CN)]4}}{Pd(CN)2}{1}{\ce{Pd(CN)2}的结构}{\ce{CN-}桥连的平面四方配合物}
在\ce{CuCN.NH3}中有\ce{CN-}桥连的二维结构(除去对\ce{Cu}配位的\ce{NH3}),但与上面不同的是\ce{C}同时对两个\ce{Cu^I}配位,同时这两个\ce{Cu^I}之间还有类似亲金作用的亲铜作用.这一层状结构如下所示.
\chemfig{CuCN}{1}{\ce{CuCN.NH3}的二维层}
\paragraph{氰的卤化物\ce{XCN}}
氰与四种卤素的化合物都已经制得.\ce{ClCN}和\ce{BrCN}可以由对应的卤素与碱金属氰化物反应得到:
\begin{center}
    \ce{X2 + MCN -> MCl + XCN}
\end{center}
将干燥的\ce{Hg(CN)2}与\ce{I2}反应即可得到\ce{ICN}:
\begin{center}
    \ce{Hg(CN)2 + 2I2 -> 2ICN + HgI2}
\end{center}
而\ce{FCN}则由\ce{(FCN)3}的分解得到.后者可以由\ce{NaF}与\ce{(ClCN)3}反应得到.事实上,所有四种\ce{XCN}都可以发生三聚形成含有六元环的\ce{(XCN)3},其结构如下:
\chemfig{(XCN)3}{1}{\ce{(XCN)3}的结构}
\ce{XCN}可以发生取代反应,得到各种含有氰基的化合物.以下是一些例子:
\begin{center}
    \ce{XCN + NaN3 -> N3CN + NaX}\\
    \ce{XCN + 2NaOH + NaX + NaOCN + H2O}\\
    \ce{ClCN + AgOCN -> NCOCN + AgCl}
\end{center}
\paragraph{氨基氰\ce{H2NCN}及其衍生物}
\subparagraph{氨基氰的制备与结构}
氨基氰(即氰胺)可以由氰氨化钙按如下方式制备得到:
\begin{center}
    \ce{2CaCN2 + 2H2O -> Ca(HNCN)2 + Ca(OH)2}\\
    \ce{Ca(HNCN)2 + CO2 + H2O -> CaCO3 + 2H2NCN}
\end{center}
氰胺是平面型分子,其结构如下:
\chemfig{H2NCN}{1}{氰胺分子的结构}
氰胺也有一个不稳定的异构体,即二亚胺\ce{HN=C=NH}.有机化学中常用的脱水剂DCC(即二环己基二亚胺)就是二亚胺的衍生物.
\subparagraph{氨基氰的反应}
\indent 常用的氰胺是以二聚体的形式存在的.这二聚体也有一个异构体,如下所示.
\bichemfig{(H2NCN)2-1}{1}{\ce{H2NCN}的二聚体I}{(H2NCN)2-2}{1}{\ce{H2NCN}的二聚体II}{\ce{H2NCN}的二聚体}
氰胺可以发生三聚.三聚氰胺常用作制备树脂,用作阻燃剂,减水剂和甲醛清洁剂等等.不法商家曾经将三聚氰胺加入奶粉中以提高氮含量,进而让检测出的蛋白质含量更高\footnote{这就是曾经轰动全国的著名的三鹿奶粉事件.}.由于三聚氰胺对神经系统和泌尿系统的毒害作用,这对婴幼儿有极其恶劣的健康影响.
\chemfig{(H2NCN)3}{1}{三聚氰胺的结构}
氰胺在酸性溶液中可以和\ce{HNO2}反应,产生异氰酸:
\begin{center}
    \ce{H2NCN + HNO2 -> HNCO + N2 + H2O}
\end{center}
氰胺也可以和尿素发生加和反应:
\begin{center}
    \ce{H2NCN + (NH)2CO -> H2N-C(NH)-NH-CO-NH2}
\end{center}
\paragraph{氰酸HOCN,异氰酸HNCO和雷酸HONC}
\subparagraph{\ce{HOCN},\ce{HNCO}和\ce{HONC}的结构}
氰酸和异氰酸互为互变异构体.除此之外,它们还有一种键合异构体,即雷酸.它们的结构如下所示:
\begin{figure}[H]
    \centering
    \subfigure[氰酸\ce{HOCN}的结构]{
        \begin{minipage}[b]{.3\linewidth}
            \centering\includegraphics{picture/HOCN.eps}
        \end{minipage}
    }
    \subfigure[异氰酸\ce{HNCO}的结构]{
        \begin{minipage}[b]{.3\linewidth}
            \centering\includegraphics{picture/HNCO.eps}
        \end{minipage}
    }
    \subfigure[雷酸\ce{HCNO}的结构]{
        \begin{minipage}[b]{.3\linewidth}
            \centering\includegraphics{picture/HCNO.eps}
        \end{minipage}
    }\caption{氰酸,异氰酸和雷酸的结构}
\end{figure}
和氰的卤化物一样,氰酸和异氰酸也有其三聚体.大多数地方都将其画成与\ce{(XCN)3}一样的具有三嗪结构的环,但光谱和晶体衍射数据都表明三聚体事实上绝大部分以三酮的形式存在.在溶液中,也少量地存在单羟基的形式.这可能是酰胺结构的强稳定性导致的.
\bichemfig{(HOCN)3}{1}{三聚氰酸可能存在的形式}{(HNCO)3}{1}{三聚氰酸实际上的存在形式}{三聚氰酸的结构}
\subparagraph{\ce{HNCO},\ce{HONC}及其三聚体的性质}
一个容易引起混淆的问题是氰酸和异氰酸的稳定性.尽管从命名上说,带异字的异构体通常是不稳定的,但事实上异氰酸才是更稳定的异构体.两者之间具有明显的能量差距\footnote{\textit{J. Chem. Phys.}  \textit{120}, 11586(\tbf{2004}); doi: 10.1063/1.1707013.}:
\begin{center}
    \ce{HOCN <=> HNCO}\ \ \ $\Delta H^{\ominus}(298\K)=-103.4\kJm$
\end{center}

\indent 加热分解三聚氰酸可以得到这两种异构体,但绝大部分为\ce{HNCO}.酸化\ce{OCN-}的盐也将得到异氰酸.将尿素加热可以按照以下方法制得异氰酸:
\begin{center}
    \ce{CO(NH2)2 ->T[$\Delta$] HNCO + NH3}
\end{center}

\indent 1828年,F. Wöhler报道了首例从无机物合成有机物尿素的办法,即异氰酸铵的热异构化:
\begin{center}
    \ce{NH4Cl + AgNCO -> NH4NCO + AgCl}\\
    \ce{NH4NCO ->T[$\Delta$] CO(NH2)2}
\end{center}
两者在溶液中处于平衡,大约有$5\%$以\ce{NH4NCO}的形式存在.这是首例以无机物合成有机物的反应,它打破了当时盛行的生命力论,证明了有机化合物并不只能由生命过程得到,有机化学从此开始蓬勃发展.\\
\indent 异氰酸在水中发生水解,其中可能经过了\ce{H2NCOOH}这一中间体:
\begin{center}
    \ce{HNCO + H2O -> [NH2COOH] -> NH3 + CO2}
\end{center}
\indent 三聚氰酸和三聚氰胺一样,本身的毒性是较低的.然而,混合两者可以形成不溶的,具有二维层状结构的氰尿酸三聚氰胺,其结构如下:
\chemfig{(HNCO)3(H2NCN)3}{1}{氰尿酸三聚氰胺的层状结构}
这是超分子组装的一个经典例子.单独摄入三聚氰酸对人体没有明显的毒害作用,但单独摄入三聚氰胺或两者一起摄入时(三聚氰胺在人体中也可以水解为三聚氰酸),由于人体无法利用这两种物质,因此它们将被运往肾脏并准备排出体外.由于肾脏的浓缩作用,这两种物质在血液中的含量增加,超过临界浓度,在肾小管中形成大量黄色粒状的氰尿酸三聚氰胺固体,致使肾小管的物理阻塞,尿液无法排出,从而最终导致肾脏衰竭.\\
,
\end{document}