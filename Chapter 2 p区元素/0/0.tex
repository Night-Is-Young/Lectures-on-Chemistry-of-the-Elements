\documentclass{ctexart}
\usepackage{EC}
\title{\tbf{0族元素}}
\author{夜未央}
\begin{document}
\maketitle
\newpage
\section{0族元素}
\subsection{0族元素综述}
\subsubsection{0族元素的发现与命名}
我们从元素的希腊文开始介绍\footnote{笔者并不是要求读者掌握希腊文的意思,只是作为繁忙的学习中的一点调剂.}.
\begin{table}[H]
    \centering
    \begin{tabular}{|c|c|c|c|c|}
        \hline
        中文名称&元素符号&英文名称&希腊文&含义\\\hline
        氦&\ce{He}&Helium&$\acute{\eta}\lambda\iota o \varsigma$&太阳\\\hline
        氖&\ce{Ne}&Neon&$\nu\acute{\ep}o\nu$&新\\\hline
        氩&\ce{Ar}&Argon&$\alpha\rho\gamma\acute{o}\nu$&懒惰\\\hline
        氪&\ce{Kr}&Krypton&$\kappa\rho\nu\pi \tau\acute{o}\nu$&隐蔽\\\hline
        氙&\ce{Xe}&Xenon&$\zeta\acute{\ep}\nu o\nu$&奇异\\\hline
        氡&\ce{Rn}&Radon&radius\footnotemark&射线\\\hline
    \end{tabular}
\end{table}\footnotetext{显然地,这是拉丁文而非希腊文.}
稀有气体单质由于其反应惰性(\ce{Ar}得名于此),因此直到1785年才被发现.首个发现的即\ce{Ne}(它也得名于此),此后其它的稀有气体单质才陆续地被发现.\\
\indent 在很长的一段时间内,人们都认为由于其满壳层的电子结构带来的稳定性使得它们不参与化学反应,因此很长一段时间内它们都被称作“惰性气体”,然而后面的研究推翻了这一点.因此称它们为“稀有气体”更加合适,因为空气中的稀有气体单质含量较低.
\subsubsection{0族元素的物理性质}
所有稀有气体单质都是无色无味的无毒气体.由于这些物质在组成分子上的相似性,因此其物理性质呈现很好的递变规律.
\begin{table}[H]
    \centering
    \begin{tabular}{|c|c|c|c|c|}
        \hline
        物质&第一电离能$/\kJm$&熔点/K&沸点/K&蒸发焓$/\kJm$\\\hline
        \ce{He}&$2372$&$-$\footnotemark&$4.215$&$0.08$\\\hline
        \ce{Ne}&$2080$&$24.55$&$27.07$&$1.74$\\\hline
        \ce{Ar}&$1520$&$83.78$&$87.29$&$6.52$\\\hline
        \ce{Kr}&$1351$&$115.9$&$119.7$&$9.05$\\\hline
        \ce{Xe}&$1170$&$161.30$&$165.04$&$12.65$\\\hline
        \ce{Rn}&$1037$&$83.78$&$87.29$&$9.05$\\\hline
    \end{tabular}
\end{table}\footnotetext{{对于\ce{He}而言,常压下并不能靠降温使其凝固,只能通过加压办到.\ce{He}也是少数没有固-液-气三相点的物质.}}
原子间的范德华力随着极化率的增大而增大,这在蒸发焓和沸点上有着直接的体现.\\
\indent 稀有气体原子的电离能在同周期是最高的.然而,较重的稀有气体元素的电离能却小于\ce{O}和\ce{F}等元素,这也暗示了它们的化合物的形成.我们将在后面重点提及这些化合物.\\
\indent 作为这一族中不寻常的一个,\ce{He}(准确而言是\ce{^4He})在接近$0\K$时有着神奇的物相变化.在大约$2.2\K$时,\ce{He}会从正常的液体(\ce{He(I)}相)转变为超流体\footnote{由于\ce{^4He}是自旋为零的Bose子,因此会发生Bose-Einstein凝聚.这涉及到二级相变的内容,并不要求掌握.}(\ce{He(II)}相).超流体\ce{^4He}具有异常高的热导率和比热容,其动力粘度为零.这一流体最值得称道的性质是其超流性.\ce{He(II)}会沿着置于其中的空杯的杯壁爬升并越过杯口进入容器中,直到内外液面相平.
\subsection{0族元素的分布和应用}
\subsubsection{\ce{He}}
\paragraph{\ce{He}的分布}
宇宙中的\ce{He}是第二丰富的元素.然而由于其过低的密度,因此不能被地球的重力场所维系,因此地球上的\ce{He}较少,主要的是\ce{^4He},来源于各种重元素的$\alpha$衰变.\\
\indent 然而,月球上却有着远比地球多得多的\ce{^3He}储备.这些\ce{^3He}主要来源于太阳风,估计有约100万吨的\ce{^3He}吸附在月球表面的硅酸盐矿物的晶格孔隙中.而地球由于磁场的保护,太阳风并不能长驱直入地击中地表,因此地球上的\ce{^3He}较为少见.
\paragraph{\ce{He}的用途}
\indent \ce{He}的最大的用途是作为各种工业或实验过程的保护气.例如作为加压和吹扫气体,电弧焊接时的保护气体,及参与制造芯片的化学反应过程.日常生活中的小用途则是作为气球或飞艇上升所需的气体.在密度较小而安全的气体中几乎只有这一选择(充\ce{H2}的气艇在历史上发生过不止一起爆炸事件.)\\
\indent 液氦由于其较低的沸点($-268.93\K$)可以用于冷却剂,研究各种超导物质.\\
\indent 由于氦在血液中的溶解度很低,因此可以加到氧气中防止减压病,作为潜水员的呼吸用气体,或用于治疗气喘和窒息.\\
\indent 与密度和空气密度不同的任何气体一样,吸入少量氦气会暂时改变人声的音调.
\subsubsection{其余0族元素}
其余0族元素的主要应用是霓虹灯中的保护气.尽管它们确实显现不同的颜色,但一般产生鲜艳的颜色仍然需要额外加入荧光物质.\\
\indent 除去产量极低的\ce{Ne,Kr}和\ce{Xe}以外,\ce{Ar}也常作为保护气使用.
\subsection{Xe的化学性质}
\subsubsection{\ce{Xe}的首个化合物}
自N. Barlett发现\ce{Xe}的电离能小于\ce{O2},受到启发而制备得到\ce{XePtF6}以来,稀有气体化合物的族谱在化学史上徐徐展开.将\ce{PtF6}蒸汽与等量的\ce{Xe}混合即可得到橙黄色固体“\ce{XePtF6}”,制备反应的方程式可以写作:
\begin{center}
    \ce{Xe + PtF6 -> XePtF6}
\end{center}
\indent 研究表明\footnote{L. Graham, O. Graudejus, N. K. Jha, N. Bartlett, Coord. Chem. Rev. 197 (2000) 321, https://doi.org/10.1016/S0010-8545(99)00190-3},\ce{XePtF6}实际上并非\ce{[Xe]+[PtF6]-},而含有\ce{[XeF]+[PtF6]-},\ce{[XeF]+[Pt2F11]-},\ce{[Xe2F3]+[PtF6]-}等化合物.这一反应并不像开始看起来那样简单,而且计量关系也很可能比较复杂.\\
\indent \ce{XePtF6}具有强氧化性,遇水迅速分解:
\begin{center}
    \ce{2XePtF6 + 6H2O -> 2Xe + O2 + 2PtO2 + 12HF}
\end{center}
在此之后,人们又制备出了许多\ce{Xe(MF6)_x}形式的化合物,并最终制备得到了\ce{Xe}的氟化物.
\subsubsection{\ce{Xe}的氟化物}
\paragraph{\ce{Xe-F}平衡体系}
\ce{Xe}和\ce{F2}在加热下仅生成\ce{XeF2},\ce{XeF4}和\ce{XeF6}.这也是\ce{Xe}仅有的三种氟化物.它们总是处于平衡之中:
\begin{center}
    \ce{Xe + F2 <=> XeF2}\ \ \ $K_1(774.2\K)=29.8$\\
    \ce{Xe + 2F2 <=> XeF4}\ \ \ $K_2(774.2\K)=0.50$\\
    \ce{Xe + 3F2 <=> XeF6}\ \ \ $K_3(774.2\K)=3.3\times10^{-3}$
\end{center}
这一体系有时也用于平衡计算的练习.总之,控制温度,投料比等条件可以选择性的合成比较纯的\ce{XeF_n}$(n=2,4,6)$.
\paragraph{\ce{Xe}的氟化物的结构}
我们可以很容易地根据VSEPR理论画出上述三种氟化物的结构.
\paragraph{\ce{XeF2}}
将\ce{Xe}和\ce{F2}混合后光照,或者将\ce{F2}与过量的\ce{Xe}混合后在\ce{Ni}制容器中加热即可得到\ce{XeF2}.
\begin{substance}[\ce{XeF2}]
    二氟化氙,化学式为\ce{XeF2},是无色固体,室温下易升华.气态\ce{XeF2}具有令人作呕的恶臭味.
\end{substance}
固体\ce{XeF2}属于四方晶系.
\chemfig{XeF2}{0.125}{\ce{XeF2}的晶体结构}
同其它氟化氙相比,\ce{XeF2}是稍温和一些的氟化剂和氧化剂.因此,它经常被用作氟化剂.\\
\indent \ce{XeF2}也可以与某些酸(主要是各非金属的最高价含氧酸或衍生物)反应生成\ce{Xe(OR)2}.例如:
\begin{center}
    \ce{XeF2 + HOSO2F -> FXe(OSO2F) + HF}\\
    \ce{FXe(OSO2F) + HOSO2F -> Xe(OSO2F)2 + HF}
\end{center}
\ce{Xe(OSO2F)}和\ce{FXe(OSO2F)}对热不稳定,容易分解:
\begin{center}
    \ce{Xe(OSO2F)2 ->T[$\Delta$] Xe + S2O6F2}\\
    \ce{2FXe(OSO2F) ->T[$\Delta$] Xe + XeF2 + S2O6F2}
\end{center}

\indent 高溴酸及其盐的制备一直以来是一个难题.最初人们采取核反应的方式由\ce{SeO4^2-}制得\ce{BrO4-}:
\begin{center}
    \ce{^83SeO4^2- -> ^83BrO4- + e-}
\end{center}
后来,人们首次使用\ce{XeF2}氧化\ce{BrO3^-}制得了\ce{BrO4-}:
\begin{center}
    \ce{BrO3- + XeF2 + H2O -> BrO4- + Xe + 2HF}
\end{center}

\indent \ce{XeF2}的水解也是值得探究的一个问题.总体的方程式如下:
\begin{center}
    \ce{2XeF2 + 2H2O -> 2Xe + O2 + 4HF}
\end{center}
这一反应的机理可以表示如下:
\begin{center}
    \ce{XeF2 + H2O -> XeO + 2HF}\\
    \ce{XeO + H2O -> Xe + H2O2}\\
    \ce{H2O2 + XeF2 -> Xe + O2 + 2HF}
\end{center}
在碱或金属离子的催化下这一反应将被大大加速.
\paragraph{\ce{XeF4}}
将\ce{Xe}与\ce{F2}以1:5的比例混合后在\ce{Ni}制容器中加热即可得到\ce{XeF4}.
\begin{substance}[\ce{XeF4}]
    四氟化氙,化学式为\ce{XeF4},是无色固体,室温下易升华.
\end{substance}
\ce{XeF4}的水解是比较复杂的.总反应方程式可以表示如下:
\begin{center}
    \ce{6XeF4 + 12H2O -> 2XeO3 + 4Xe + 3O2 + 24HF}
\end{center}
反应的过程大致可以表示如下:
\begin{center}
    \ce{3Xe^{IV} ->T[\ce{H2O}] 2Xe^{II} + Xe^{VIII}}\\
    \ce{Xe^{II} ->T[\ce{H2O}] Xe + O2}\\
    \ce{Xe^{VIII} ->T[\ce{H3O+}] Xe^{VI} + O2}
\end{center}
我们将在讲到\ce{XeO3}时再次提及这一反应.\\
\indent 和\ce{XeF2}一样,\ce{XeF4}也可以作为氟化剂,但氧化性更强.
\paragraph{\ce{XeF6}}
将\ce{Xe}与\ce{F2}以1:20的比例混合后在\ce{Ni}制容器中加热即可得到\ce{XeF4}.此外,用\ce{O2F2}氧化\ce{XeF4}也是可行的办法.
\begin{substance}[\ce{XeF6}]
    六氟化氙,化学式为\ce{XeF6},是无色固体,室温下易升华,其蒸汽显现黄色.
\end{substance}
\ce{XeF6}与等量水反应生成\ce{XeOF4},完全水解则生成\ce{XeO3}:
\begin{center}
    \ce{XeF6 + H2O -> XeOF4 + 2HF}\\
    \ce{XeF6 + 3H2O -> XeO3 + 6HF}
\end{center}
和\ce{SiO2}的反应也是类似的.\\
\indent 尽管\ce{XeF6}也可以氟化其他物质,但由于过于猛烈的氧化性而总是导致底物的裂解,因此很少用它作氟化剂.
\subsubsection{\ce{Xe}的氧化物含氧阴离子}
\paragraph{\ce{XeO3}与\ce{HXeO4-}}
前面已经说过,\ce{XeO3}可以由\ce{XeF4}或\ce{XeF6}的水解得到,但最好还是用\ce{XeF6}为宜.
\begin{substance}[\ce{XeO3}]
    三氧化氙,化学式为\ce{XeO3},是一种无色透明的吸湿性晶体.
\end{substance}
\ce{XeO3}具有猛烈的爆炸性.稍有摩擦或撞击,或在潮湿的环境中就容易发生爆炸(同时伴随着蓝色的闪光):
\begin{center}
    \ce{2XeO3 -> 2Xe + 3O2}
\end{center}
在碱性溶液中,\ce{Xe^{VI}}主要以\ce{HXeO4-}的形式存在,并更容易发生分解.
\paragraph{\ce{XeO6^4-}与\ce{XeO4}}
前面说到\ce{HXeO4-}在碱性条件下容易发生分解,事实上发生了如下的歧化反应:
\begin{center}
    \ce{2HXeO4- + 2OH- -> XeO6^4- + Xe + O2 + H2O}
\end{center}
因此,\ce{XeF6}在碱性条件下的水解也能获得类似的结果:
\begin{center}
    \ce{2XeF6 + 16OH- -> XeO6^4- + Xe + O2 + 12F- + 8H2O}
\end{center}
将\ce{Ba2XeO6}加入浓\ce{H2SO4}中即可制得\ce{XeO4}.这是一种极不稳定的四面体形分子,在熔融时即发生爆轰.
\subsection{其它0族元素的化学性质}
\ce{Kr}也能形成与\ce{XeF2}类似的\ce{KrF2}.几十年来,所有0族元素都已经制备得到了化合物(尽管大部分都很不稳定).唯独\ce{He}难以形成化合物.几年前合成的\ce{Na2He}(还出现在了初赛试题中)实际上写成\ce{(Na+)2(e-)2He}更为合适,其中的\ce{He}似乎并没有形成化合物的倾向.与其认为这是\ce{He}的化合物,不如说是\ce{He}吸附在晶格中的加合物.因此,0族元素的性质与反应仍然有很多值得探究之处.
\end{document}