\documentclass{ctexart}
\usepackage{EC}
\begin{document}
\section{磷及其化合物}
\subsection{单质磷}
\subsubsection{磷的同素异形体}
\paragraph{白磷}
白磷是磷最常见的单质之一,由\ce{P4}分子构成.白磷可以由液态或气态的磷蒸汽冷却得到.
\begin{substance}[\ce{P4}]
    白磷,又称为黄磷,化学式为\ce{P4},为白色的质地较软的蜡状固体,有剧毒,熔点为$44.2\tc$,沸点为$280.5\tc$.白磷难溶于水,但易溶于\ce{CS2}等有机溶剂中.
\end{substance}
与其常见性相反的是,白磷是磷单质中热力学上最不稳定的一个.白磷的一个特殊反应就是在空气中的自动氧化,这一反应发出磷光(即鬼火的来源).\\
\indent 白磷具有独特的正四面体结构.理论上其中的\ce{P-P}键的轨道重叠并非完全沿着键轴,而是有所弯曲,形成“香蕉键”.高环张力和较弱的键是白磷的高反应性的主要来源.
\chemfig{P4}{1}{\ce{P4}分子的结构}
\paragraph{红磷}
在$260\tc\sim300\tc$加热白磷即可得到红磷.红磷是复杂的高分子化合物.最早认为其中的\ce{P4}笼被部分地打开而形成长链结构,示意如下:
\chemfig{redP}{1}{曾经认为的红磷的结构示意图}
后来,又有各种链状结构被提出.直到近年来,人们认为红磷中的一维长链事实上具有如下的结构\footnote{
Zhang S. \textit{Angew. Chem. Int. Ed.} \tbf{2019}, \textit{58 (6)}, $1659-1663$.
DOI:10.1002/anie.201811152}:
\chemfig{redP-true}{1}{红磷的真实结构示意图}

\paragraph{紫磷}
紫磷可通过把白磷以$500\tc$溶解在盛有熔融的铅的密封管中18小时制得.紫磷又称Hittorf磷,其具有复杂的管状结构,示意如下:
\chemfig{purpleP}{1}{紫磷的结构示意图}
\paragraph{黑磷}
黑磷是单质热力学最稳定的形式,已制得其三种晶体和一种无定形体.它比红磷有更高的聚合度,并且其相应的密度较高.\\
\indent 正交晶型的黑磷最初是将白磷在$12000\text{ atm}$压力下加热到$200\tc$而
制得,其晶体结构如下.
\chemfig{blackP-oP}{0.125}{正交黑磷的晶体结构示意图}
\noindent 你可以把上述晶体画成下面的层状结构.
\chemfig{blackP}{1}{正交黑磷的结构示意图}
\noindent 其它晶型的黑磷也有类似的二维层状结构,只是构象有所不同.
\subsubsection{单质磷的生产和应用}
发现磷元素之后的很长一段时间内,磷的唯一来源是尿.由于尿中含有总量可观的磷酸盐,因此炼金术士们用木炭就能将其还原为\ce{P4}.现在所用的把磷酸盐矿石和砂子,焦炭一起加热来制取磷的方法是在1867年提出的,总的反应方程式可以表示如下:
\begin{center}
    \ce{2Ca3(PO4)2 + 6SiO2 + 10C -> 6CaSiO3 + 10CO + P4}
\end{center}
这一过程主要有两个副反应.首先,由于磷酸盐矿石中通常含有氟磷灰石\ce{Ca5(PO4)3F},因此可能发生下面的反应:
\begin{center}
    \ce{4Ca5(PO4)3F + 21SiO2 + 30C -> 20CaSiO3 + SiF4 + 3P4 + 30CO}
\end{center}
产生的\ce{SiF4}有毒且有腐蚀性.另外,矿物中的\ce{Fe2O3}也可能发生下面的反应:
\begin{center}
    \ce{4Fe2O3 + P4 + 12C -> 4Fe2P + 12CO}
\end{center}
生成的\ce{Fe2P}在反应条件下为粘稠的液体,沉在反应炉的底部而难以排出.
\subsubsection{单质磷的反应}
磷几乎与所有元素都能形成化合物.相关的反应将在接下来几节介绍.这里只介绍一个经典的,也是计量系数相当复杂的一个反应,即\ce{P4}与\ce{CuSO4}溶液的反应:
\begin{center}
    \ce{11P4 + 60CuSO4 + 96H2O -> 20Cu3P + 24H3PO4 + 60H2SO4}
\end{center}
过去曾经使用这种办法来治疗急性白磷中毒.不过由于\ce{CuSO4}对肾脏和大脑的损害,这种方法已经不再使用.
\subsubsection{多磷阳离子}
对于磷元素,质谱可以在气相中表征阳离子\ce{[Pn]+}$(n=2\sim24)$,截至目前核数最多的\ce{[P91]+}也可以在气相中存在.然而,凝聚相分离表征\ce{[Pn]+}却极具挑战性.2022年8月,弗莱堡大学的Ingo Krossing教授课题组和德累斯顿工业大学的Jan Weigand教授课题组对这一反应体系进行了深入的研究,并且成功的获得了\ce{[P9]+}阳离子的晶体结构\footnote{Frötschel-Rittmeyer, J.; Holthausen, M.; Friedmann, C.; Röhner, D.;Krossing,  I.; Weigand, J. J. Homoatomiccations: From \ce{[P5]+} to \ce{[P9]+}. \textit{Sci. Adv.} \tbf{2022}, \textit{8 (36)}, No. eabq8613. DOI: 10.1126/sciadv.abq8613}.制备的反应方程式如下:
\begin{center}
    \ce{P4 + PCl3 + 2GaCl3 -> [P5Cl2]+[Ga2Cl7]-}\\
    \ce{[P5Cl2]+[Ga2Cl7]- + Ga3Cl7 -> [P5Ga2Cl6]+[Ga2Cl7]- + GaCl3}\\
    \ce{[P5Ga2Cl6]+[Ga2Cl7]- + P4 -> [P9]+[Ga2Cl7]- + 2GaCl3}
\end{center}
这一过程中的含\ce{P}离子的结构如下(包括重要的反应中间体\ce{[P5]+}):

\subsection{磷的氢化物}
\subsubsection{\ce{PH3}}
\begin{substance}[\ce{PH3}]
    磷化氢又名膦,化学式为\ce{PH3},是无色无味(也有说法认为有微弱大蒜气味),可燃且剧毒的气体,熔点为$-132.8\tc$,沸点为$-87.7\tc$.\\
    纯的\ce{PH3}无臭,但工业生产的\ce{PH3}含有\ce{P2H4}等杂质,因此有很臭的腐烂鱼腥味.同样,含有微量\ce{P2H4}的\ce{PH3}在空气中即发生自燃.
\end{substance}
\paragraph{\ce{PH3}的结构}
这是一个值得探讨的问题.\\
\indent 我们已经知道\ce{PH3}的键角为$93.5^\circ$,与\ce{AsH3}和\ce{SbH3}相似,而与\ce{NH3}的$107.8^\circ$相去甚远.从杂化论,我们一般可以解释为第三周期及以后的元素的p轨道与s轨道能量差距更大,孤对电子倾向占据s轨道,杂化的程度小,因此键角接近$90^\circ$.\\
\indent 然而,
\subsubsection{\ce{P2H4}}
\subsection{磷的卤化物}
\subsubsection{\ce{PX3}}
\subsubsection{\ce{PX5}}
\subsubsection{\ce{P2X4}}
\subsection{磷的卤氧化物}
\subsection{磷的氧化物和硫化物}
\subsubsection{磷的氧化物}
\subsubsection{磷的硫化物}
\subsection{磷的含氧酸}
\subsubsection{次磷酸及其盐}
\subsubsection{亚磷酸及其盐}
\subsubsection{连二磷酸及其盐}
\subsubsection{磷酸及其盐}
\subsection{磷氮化合物}
\end{document}