\documentclass{ctexart}
\usepackage{EC}
\begin{document}
\section{磷及其化合物}
\subsection{单质磷}
\subsubsection{磷的同素异形体}
\paragraph{白磷\ce{P4}}
这是磷最常见的单质之一,由液态或气态的磷蒸汽冷却得到.
\begin{substance}[\ce{P4}]
    白磷,又称为黄磷,化学式为\ce{P4},为白色的质地较软的蜡状固体,有剧毒,熔点为$44.2\tc$,沸点为$280.5\tc$.白磷难溶于水,但易溶于\ce{CS2}等有机溶剂中.
\end{substance}
与其常见性相反的是,白磷是磷单质中热力学上最不稳定的一个.白磷的一个特殊反应就是在空气中的自动氧化,这一反应发出磷光(即鬼火的来源).\\
\indent 白磷具有独特的正四面体结构.理论上其中的\ce{P-P}键的轨道重叠并非完全沿着键轴,而是有所弯曲,形成“香蕉键”.高环张力和较弱的键是白磷的高反应性的主要来源.
\chemfig{P4}{\ce{P4}分子的结构}
\paragraph{红磷}
在$260\tc\sim300\tc$加热白磷即可得到红磷,其中的\ce{P4}笼被部分地打开而形成长链结构,示意如下:
\chemfig{redP}{红磷的结构示意图}
\paragraph{紫磷}
紫磷可通过把白磷以$500\tc$溶解在盛有熔融的铅的密封管中18小时制得.紫磷又称Hittorf磷,其具有复杂的管状结构,示意如下:
\chemfig{purpleP}{紫磷的结构示意图}

\subsubsection{单质磷的生产和应用}
发现磷元素之后的很长一段时间内,磷的唯一来源是尿.由于尿中含有总量可观的磷酸盐,因此炼金术士们用木炭就能将其还原为\ce{P4}.现在所用的把磷酸盐矿石和砂子,焦炭一起加热来制取磷的方法是在1867年提出的,总的反应方程式可以表示如下:
\begin{center}
    \ce{2Ca3(PO4)2 + 6SiO2 + 10C -> 6CaSiO3 + 10CO + P4}
\end{center}
这一过程主要有两个副反应.首先,由于磷酸盐矿石中通常含有氟磷灰石\ce{Ca5(PO4)3F},因此可能发生下面的反应:
\begin{center}
    \ce{4Ca5(PO4)3F + 21SiO2 + 30C -> 20CaSiO3 + SiF4 + 3P4 + 30CO}
\end{center}
产生的\ce{SiF4}有毒且有腐蚀性.另外,矿物中的\ce{Fe2O3}也可能发生下面的反应:
\begin{center}
    \ce{4Fe2O3 + P4 + 12C -> 4Fe2P + 12CO}
\end{center}
生成的\ce{Fe2P}在反应条件下为粘稠的液体,沉在反应炉的底部而难以排出.
\subsubsection{单质磷的反应}
\subsection{多磷阳离子}
\subsection{磷化物}
\subsection{磷的氢化物}
\subsubsection{\ce{PH3}}
\subsubsection{\ce{P2H4}}
\subsection{磷的卤化物}
\subsubsection{\ce{PX3}}
\subsubsection{\ce{PX5}}
\subsubsection{\ce{P2X4}}
\subsection{磷的卤氧化物}
\subsection{磷的氧化物和硫化物}
\subsubsection{磷的氧化物}
\subsubsection{磷的硫化物}
\subsection{磷的含氧酸}
\subsubsection{次磷酸及其盐}
\subsubsection{亚磷酸及其盐}
\subsubsection{连二磷酸及其盐}
\subsubsection{磷酸及其盐}
\subsection{磷氮化合物}
\end{document}