\documentclass{ctexart}
\usepackage{EC}
\begin{document}
\section{氮及其化合物}
\subsection{单质氮}
\ce{N2}几乎是氮唯一稳定的单质.
\begin{substance}[\ce{N2}]
    氮气,化学式为\ce{N2},是一种无色无味无臭的反磁性气体,熔点为$-210\tc$,沸点为$-195.8\tc$.
\end{substance}
\subsubsection{\ce{N2}的化学性质}
\ce{N2}在常温下是相当不活泼的,这可能是因为
\begin{enumerate}[label=\tbf{\arabic*.},topsep=0pt,parsep=0pt,itemsep=0pt,partopsep=0pt]
    \item \ce{N2}具有很大的键能(键解离能为$945.41\kJm$).
    \item \ce{N2}的HOMO-LUMO能级间距大.
    \item \ce{N#N}键的电子分布非常均匀,没有极性.
\end{enumerate}
\ce{N2}的反应性随温度升高而增加.\ce{N2}能与活泼金属反应生成氮化物,与\ce{H2}反应\footnote{这是化工行业重要的反应之一,也常常出现于各类平衡计算中.}生成\ce{NH3},与焦炭反应生成\ce{(CN)2}:
\begin{center}
    \ce{2C + N2 ->T[$\Delta$] (CN)2}
\end{center}
虽然\ce{N2}相比\ce{CO}也是一个较差的配体,但仍然能形成各种配合物.最早制得的\ce{N2}配合物为\ce{[Ru(NH3)5(N2)]^2+},是用水合肼还原\ce{RuCl3}水溶液而制得的.\ce{N2}的配合物是无机化学研究的重要领域,这与固氮有着重要的关联.
\subsubsection{\ce{N2}的生产与用途}
大规模制取\ce{N2}的唯一途径是分馏空气(这和制取\ce{O2}是一致的),实验室也多采取氮气瓶供给的方式实现.其它的可行的办法包括:
\begin{center}
    \ce{2NaN3 ->T[$300\tc$] 2Na + 3N2}\\
    \ce{NH4NO2(aq) ->T[$\Delta$] N2(g) + 2H2O(l)}\\
    \ce{(NH4)2Cr2O7 -> N2 + 2Cr2O3 + 4H2O}\\
    \ce{8NH3 + 3Br2 ->T[aq] N2 + 6NH4Br}\\
    \ce{2NH3 + 3CuO -> N2 + 3Cu + 3H2O}
\end{center}
\subsection{全氮离子}
我们主要介绍两种全氮离子:\ce{N5+}和\ce{N5-}.巧合的是,这两种离子分别由美国和中国的研究团队发现,并且都是研发高能炸药时制得的.
\subsubsection{全氮阳离子}
\ce{N5+}唯一的制备方法如下:
\begin{center}
    \ce{cis-N2F2 + SbF5 ->T[$-78\tc$] [N2F]+[SbF6]-}\\
    \ce{ [N2F]+[SbF6]- + HN3 ->T[$-78\tc$][\ce{HF}] [N5]+[SbF6]- + HF}
\end{center}
\ce{N5+}的结构示意如下.
\chemfig{N5+}{\ce{N5+}的结构}
你可以尝试画出它的共振式以验证这一结构的合理性.\\
\indent 值得一提的是,\ce{N5+}可以形成诸如\ce{[N5]+[P(N3)6]-}和\ce{[N5]+[B(N3)4]-}的离子.它们也许是含\ce{N}量较高的化合物之一.
\subsubsection{全氮阴离子}
2017年,中国南京理工大学化工学院胡炳成教授课题组首次合成出室温下稳定的\ce{N5-}盐,发表于Science杂志上.
\chemfig{N5-}{\ce{N5-}的合成路线及其结构}
这一离子的稳定性部分源于其芳香性,它是咪唑阴离子的等电子体.笔者猜测\ce{HN5}也许是含\ce{N}量最高的化合物.\footnote{可惜的是,\ce{[N5]+[N5]-}理当不存在.否则我们就能见到一种看起来就很适合做炸药的物质了.}
\subsection{氮化物与叠氮化物}
\subsubsection{氮化物}
氮几乎和周期表中的所有元素形成二元化合物(略少于\ce{O}和\ce{F}).它们的结构通常比较复杂.金属氮化物的制备方式主要有:
\begin{enumerate}[label=\tbf{\arabic*.},topsep=0pt,parsep=0pt,itemsep=0pt,partopsep=0pt]
    \item 金属与\ce{N2}(通常在高温下)反应,例如
        \begin{center}
            \ce{3Ca + N2 -> Ca3N2}
        \end{center}
    \item 金属与\ce{NH3}(通常也在高温下)反应,例如
        \begin{center}
            \ce{3Mg + 2NH3 -> Mg3N2 + 3H2}
        \end{center}
    \item 金属氨基化合物的热分解,例如
        \begin{center}
            \ce{3Zn(NH2)2 -> Zn3N2 + 4NH3}
        \end{center}
    \item 在有还原剂存在时还原金属氧化物或卤化物,例如
        \begin{center}
            \ce{Al2O3 + 3C + N2 -> 2AlN + 3CO}\\
            \ce{2ZrCl4 + N2 + 4H2 -> 2ZrN + 8HCl}
        \end{center}
    \item 在液氨中进行的反应.
\end{enumerate}
根据性质的不同,我们可以把氮化物分成三类:离子型,共价型和金属型.
\subsubsection{叠氮化物}
\subsection{氨和铵盐}
\begin{substance}[\ce{NH3}]
    氨,化学式为\ce{NH3},是无色的具有特殊刺激性气味的气体.较高浓度的\ce{NH3}是有毒的.\ce{NH3}的熔点为$-77.7\tc$,沸点为$-33.4\tc$.
\end{substance}
\subsubsection{氨的结构与物理性质}
液态\ce{NH3}具有较低的密度和粘度,但具有很高的介电常数.\\
\indent 一个有必要探讨的话题是固态氨中的氢键数目.事实上,与\ce{H2O}不同,固态氨中的每个\ce{N}原子均形成三组氢键,因此这一氢键事实上与冰中具有芳香性和饱和性的氢键并不相同,更像是一种静电相互作用.这也导致了固态氨融化时实际上沉在液相的底部\footnote{参考数据:$198\K$时,液氨的密度为$0.731\text g\cdot\text{cm}^{-3}$;$193\K$时固态氨的密度为$0.820\text g\cdot\text{cm}^{-3}$.},这与冰和水恰恰相反.
\subsubsection{液氨作为一种溶剂}
液氨是人们最熟悉的,研究得最充分的非水离子化体系.在这其中可以发生许多典型的反应.
\paragraph{溶解碱金属}
碱金属最引人注意的性质之一是它们易溶于液氮并形成亮蓝色的,具有异乎寻常性质的亚稳态溶液.重碱土金属,\ce{Eu}和\ce{Yb}也有类似的性质.\\
\indent 这些溶液具有相似的性质.在浓度较低时,它们都呈现蓝色,而在浓度较高时则显现金属般的赤褐色.随着浓度增大,溶液的电导率先减小后增大,磁性也由低浓度的顺磁性变为抗磁性,最后又显现微弱的顺磁性.\\
\indent 现在,我们普遍认为其中含有金属离子\ce{M^x+}的氨配合物\ce{[M(NH3)n]^x+}和分布在空穴中的自由电子\ce{e-}.这些空穴是排开\ce{NH3}分子形成的,因此会使得溶液密度大大减小.溶液显现蓝色正是由于自由电子的存在.\\
\indent 我们可以用以下几种物质的平衡来解释金属-液氨溶液随浓度发生的性质变化\footnote{这里省略表示了\ce{NH3}的溶剂化作用.}:
\begin{center}
    \ce{M <=> M^+ + e^-}\ \ \ $K\sim10^{-2}$\\
    \ce{M + e- <=> M-}\ \ \ $K\sim10^{-3}$\\
    \ce{2M <=> M2}\ \ \ $K\sim5\times10^{3}$
\end{center}
当\ce{M}的总浓度较低时,第一个平衡占主要地位.此时,由于自由电子\ce{e-}的存在,溶液呈现顺磁性,并且由于\ce{e-}极高的离子迁移率,溶液的电导率也较高.当\ce{M}的浓度开始升高时,\ce{e-}开始与\ce{M}结合形成\ce{M-},电导率降低,同时二聚体\ce{M2}的形成也使得溶液呈现抗磁性.\\
\indent 碱金属的液氨溶液不稳定.在有杂质作为催化剂或放置的较久的情况下,这些溶液将分解为对应的氨基盐:
\begin{center}
    \ce{2Na + 2NH3 -> 2NaNH2 v + H2}
\end{center}
需要注意的是,\ce{NaNH2}微溶于液氨,因此放置的太久的\ce{Na}-\ce{NH3}溶液发生的现象为:溶液的蓝色逐渐褪去,产生白色沉淀.\\
\indent 由碱金属-液氨溶液出发也可以获得以电子作为阴离子的盐.例如:
\begin{center}
    \ce{Na + 2,2,2-crypt ->T[\ce{NH3(l)}][vap] [Na(2,2,2-crypt)]^+e^-}
\end{center}
所得的\ce{[Na(2,2,2-crypt)]^+e^-}是一种蓝黑色的顺磁性物质.\\
\indent 碱金属的液氨溶液具有很强的还原性.并且,由于这还原性是自由电子\ce{e-}体现的,因此它常常能做到对物质的单电子还原.例如:
\begin{center}
    \ce{Mn2(CO)10 + 2K ->T[\ce{NH3(l)}] 2K[Mn(CO)5]}\\
    \ce{Fe(CO)5 + 2Na ->T[\ce{NH3(l)}] Na2[Fe(CO)4] + CO}
\end{center}
等等.碱金属的液氨溶液在有机化学中也被广泛地运用于还原各种物质,例如将炔烃还原为反式烯烃,Birch还原等等.
\paragraph{液氨中的复分解反应}
大部分时候,我们可以把液氨中的反应与水溶液中的反应进行类比.然而,由于结构与性质上的差异,许多物质在水中的溶解度和在液氨中有明显的不同.典型的微溶物有\ce{NaF},\ce{NaCl}.大多数铵盐都易溶,而\ce{AgBr}等物质由于可以形成\ce{Ag(NH3)2+}配离子,也是易溶的.这就造成了某些水溶液中不能进行的反应反而能在液氨中进行:
\begin{center}
    \ce{Ba(NO3)2 + 2AgBr ->T[\ce{NH3(l)}] BaBr2 v + 2AgNO3}
\end{center}
以及与难氢氧化物和氧化物的形成类似地有
\begin{center}
    \ce{AgNO3 + KNH2 ->T[\ce{NH3(l)}] AgNH2 v + KNO3}\\
    \ce{3HgI2 + 6KNH2 ->T[\ce{NH3(l)}] Hg3N2 + 6KI + 4NH3}
\end{center}
\paragraph{液氨中的酸碱反应}
和水一样,液氨也可以发生自耦电离:
\begin{center}
    \ce{2NH3 <=> NH4+ + NH2-}\ \ \ $K=1.9\times10^{-33}$
\end{center}

\end{document}